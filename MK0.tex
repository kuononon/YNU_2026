\documentclass[a4paper,11pt]{jlreq}%日本語文書クラス

\usepackage{luatexja-fontspec}%日本語フォント指定
\setmainfont{Harano Aji Mincho}%本文フォント
\usepackage{amsmath,amssymb}%数式関連
\usepackage{unicode-math}%数式フォント指定
\usepackage[version=4]{mhchem}%化学式
\usepackage{graphicx}%画像挿入
\setmathfont{Latin Modern Math}%%数式フォント
\newcommand{\ctext}[1]{\raise0.2ex\hbox{\textcircled{\scriptsize{#1}}}}%丸囲み数字
\newcommand{\Resovist}{Resovist\textsuperscript{\textregistered}}
\usepackage{siunitx}%
\DeclareSIUnit\Oe{Oe}
\sisetup{
  range-phrase = --,
  range-units = single
}
\usepackage{subcaption}
\usepackage{float}
\usepackage[numbers]{natbib}
\graphicspath{{images/}
{images/MK0/}
{images/MK1/}
{images/MK2/}
{images/MK3/}
{images/MK4/}
{images/MK5/}
}







\begin{document}

\vspace{1pt}

\begin{center}
\vspace{20mm}
{\large\noindent 令和7年度 \\
卒業論文}\\
\vspace{60mm}
{\Huge\noindent\textbf{イメージング用トレーサーとしての}}\\
\medskip
{\Huge\noindent\textbf{GdドープMn-Zn ferriteの作製と磁気緩和特性}}\\
\vspace{\baselineskip}
\vspace{60mm}

{\Large\noindent
横浜国立大学 理工学部\\
数物・電子情報系学科 物理工学EP\\

一柳研究室\\
2264241 三浦 玖遠\\
}
\vspace{40mm}
\end{center}
\thispagestyle{empty}

\clearpage
\vspace*{\fill}
\begin{abstract}
磁性ナノ微粒子は超常磁性などの特有な磁気特性を持つ。近年では医療応用に向けて、これらの特性を活かした磁性ナノ微粒子の研究が行われている。
本研究は、Gd をドープしたMn-Zn ferriteナノ微粒子を、本研究室独自の湿式混合法を用いて作製した。組成は$\ce{Mn_{0.8-x}Zn_{0.2}Gd_{x}Fe2O4}$
を採用し、Gdドープ量を変え、さらに焼成温度を調整することで粒径を制御した。磁気粒子イメージング(MPI:Magnetic Particle Imaging)、
磁気共鳴イメージング(MRI:Magnetic Resonance Imaging)の新規造影剤の開発を行った。

XRD測定から、Gdドープ量$x≤0.08$においてスピネル構造を確認した。しかし、ドープ量$x≥0.09$では不純物相のピークが確認された。
また、角度分解能を上げピークシフトを測定すると、ピークが$x=0.08$付近まで低角側にシフトした。これは他の金属イオンと比べて、
イオン半径が大きいGdがドープされ、格子定数が増加したことを示唆している。これらの結果より、一定のドープ量を超えるとGdの固溶限界に達し、
不純物が生成されることを示した。

磁化測定から、作製したサンプルが超常磁性的挙動を示すことが確認された。また、ドープ量の増加に伴い飽和磁化$M_\text{s}$、初透磁率$μ_0$が増加することを確認した。
さらに$x=0.06$のサンプルを、粒径別に温度を\qtyrange{150}{350}{K}の範囲で変化させて、交流磁化率の温度依存性を測定した。測定結果より、
粒径\qty{14.1}{nm}のサンプルが体温(\qty{310}{K})において最大の交流磁化率実数部$\chi'$を示した。

MPIシグナルである第三高調波測定は駆動磁場コイルで交流磁場を印加し、ピックアップコイルを用いることによってサンプルからの第三高調波応答を読み取った。
測定結果より最も大きな初透磁率$μ_0$を示す、$x=0.06$のサンプルが最大の応答強度を示した。

MRI測定では、$x=0.06$のサンプルがどれも寒天より高い$T_1$緩和能$R_1$を示した。これは高い有効磁気モーメントを持つ\ce{Gd}をドープしたことに起因し、
MRIでの撮像の際にコントラストの向上を期待できる。a
\end{abstract}
\vspace*{\fill}
\clearpage


\tableofcontents

\newpage

\section{諸元}
\subsection{研究背景}
\subsubsection{ナノテクノロジー}
ナノテクノロジーとはナノスケール(\qtyrange{1}{100}{nm})での加工、操作、制御を指し、医療分野やエレクトロニクス
における応用が期待される技術である\cite{Kim2008}。当研究室では、ナノサイズ化した磁性ナノ微粒子を応用した研究が進められてきた\cite{Sakamoto2023,Aoki2022}。
磁気ナノ微粒子は特異的な磁気特性を持ち、磁気イメージングにおいて、
本研究でも扱う磁気粒子イメージング(MPI)や、磁気共鳴イメージング(MRI)などが盛んに研究されている\cite{Na2007,Gleich2005}。MPIは、GleichとWeizeneckerにより提唱された
新しいイメージング技術であり、この技術はトレーサーそのもののシグナルを読み取るため、高感度かつ高空間分解能で、磁性ナノ微粒子の分布を可視化できる\cite{Gleich2005}。
MPIに関する研究では、\Resovist を用いた研究が主流である。MRIはBlochとPurcellにより提唱された\cite{Bloch1946,Purcell1946}。
NMR現象を用いた測定であり、水素原子中のプロトン(\ce{^1H})からのシグナルにより、$T_1$強調画像や$T_2$強調画像を得られる。MRI測定では体内の
水素原子中のプロトン(\ce{^1H})を利用するが、特定の病変を画像のコントラストを上げて見るため、Gd-DTPAなどの造影剤を用いて測定されることもある\cite{Na2007}。さらに両手法において
新規造影剤の開発も盛んに行われている。

\subsection{研究目的}
イメージング用の新規造影剤として、高い磁化と低い磁気異方性を持つ、磁性ナノ微粒子が最適であると考えられている。先行研究ではMn ferriteナノ微粒子に、
\ce{Gd}をドープすることで初透磁率が増加することが報告されている\cite{Sakamoto2023}。Mn-Zn ferriteナノ微粒子に関しては、$\ce{Mn_{0.8}Zn_{0.2}Fe2O4}$の組成において、
最も大きな飽和磁化を示し、超常磁性であることが知られている\cite{Kondo2015,Kondo2014}。そこで本研究では、Mn-Zn ferriteに\ce{Gd}をドープし、粒径をナノサイズにすることで
造影剤としての性能評価を目的としている。作製した磁気ナノ微粒子は構造解析、磁気特性解析を行った。さらにアミノ基を導入する実験を行うことで、
選択性を持つ磁気ナノ微粒子としての可能性を評価した。MPI測定では、作製した磁気ナノ微粒子は、\ce{Gd}ドープ量を変えMPIシグナルの最適化を行った。MRI測定では、\ce{Gd}ドープにより水素原子中のプロトンの磁気緩和現象に
どのような影響を与えるかを評価した。以上の研究で得られた結果から、イメージング用新規造影剤としての可能性を深く検討した。



\section{理論}

\subsection{磁性}
大きさ$H$の磁場が印加されると、物質の磁化$M$は比例定数を用いて以下の等式を立てることができる。
\begin{equation}\label{磁性}
  M = \chi H
\end{equation}
ここで$\chi$は磁化率であり、この値が大きいほど磁化が大きいと評価される。

\subsubsection{強磁性}
強磁性は一般的に自発磁化を持つと知られている。図\ref{fig:ヒステリシス曲線}に示すように強磁性体の磁気特性は、ヒステリシス(履歴)を持ち消磁状態から徐々に磁場を強めていき、飽和するまで印加する。このとき$H = 0$付近の傾きは初磁化率$\chi_0$と呼ばれる。磁化が飽和した状態から磁場を反対方向に磁化が飽和するまで印加する。この時$H = 0$での磁化を残留磁化$M_{\text{r}}$、$M = 0$になった時の磁場を保磁力$H_\text{c}$、磁場$H$を大きくしていき一定になる磁化$M$は$M_\text{s}$と呼ばれる。
\begin{figure}[H]
  \centering
  \includegraphics[width=0.5\textwidth]{ヒステリシス曲線.pdf}
  \caption{ヒステリシス曲線}
  \label{fig:ヒステリシス曲線}
\end{figure}
\begin{figure}[H]
  \centering
  \includegraphics[width=0.6\textwidth]{強磁性.png}
  \caption{強磁性}
  \label{fig:強磁性}
\end{figure}

\subsubsection{反強磁性}
反強磁性は隣り合う磁性原子のスピンが、図\ref{fig:反強磁性}のように反対の方向を向きお互いのスピンによる磁化を打ち消すことである。
主に反強磁性体には\ce{MnO}などがある\cite{strong1948}。
\begin{figure}[H]
  \centering
  \includegraphics[width=0.6\textwidth]{反強磁性.png}
  \caption{反強磁性}
  \label{fig:反強磁性}
\end{figure}

\subsubsection{フェリ磁性}
フェリ磁性は反強磁性と同様にスピンが反平行になるが、各スピンの大きさに偏りが生じると、正味の磁化が現れる。\\
\begin{figure}[H]
  \centering
  \includegraphics[width=0.6\textwidth]{フェリ磁性.png}
  \caption{フェリ磁性}
  \label{fig:フェリ磁性}
\end{figure}

フェリ磁性の代表的なものとしてフェライトが知られている\cite{strong1948}。スピネル型結晶構造の単位格子は、図\ref{fig:spinell}のような構造を有しており、4つの酸素イオンが正四面体を形成する四面体サイト(A-site)が8個と、6つの酸素イオンが正八面体を形成する八面体サイト(B-site)が16個で形成されている。
フェライトの化学式は\ce{MFe2O4}で表される。\ce{M^{2+}}は2価の金属イオンであり、その種類によって異なるため、A-siteに入るものを正スピネル、B-siteに入るものを逆スピネルと分けることができる\cite{strong1948}。
この構造はフェリ磁性を持つことが知られているが、これはA-siteのスピンとB-siteのスピンが、$\ce{O^{2-}}$イオンを媒介とした超交換相互作用が働くためである。
B-site間でも超交換相互作用は働くが、その際にスピンが反平行に配列するかは、結合角度やどのサイト間の相互作用が支配的であるかによって異なる\cite{Goodenough1955,Kanamori1959,Anderson1950}。
\begin{figure}[htbp]
  \centering
  \includegraphics[width=0.7\textwidth]{スピネル構造.png}
  \caption{スピネル構造}
  \label{fig:spinell}
\end{figure}



\subsubsection{超常磁性}
超常磁性体の磁化曲線は図\ref{fig:超常磁性} (a)に示すようにヒステリシスが消失する。
これは一度磁化を飽和させても磁場を下げると、磁気異方性エネルギーが熱エネルギーを下回り、熱的揺らぎによりスピンの向きが不安定になり、平均の磁化が0になるためである。
\begin{figure}[H]
  \centering
  \includegraphics[width=0.8\textwidth]{超常磁性.png}
  \caption{(a)超常磁性体の$M$-$H$ループ(b)超常磁性のイメージ図}
  \label{fig:超常磁性}
\end{figure}

\subsubsection{超交換相互作用}
多くの物質において磁気モーメントの配列は、磁気モーメント間に働く相互作用によって引き起こされる。ひとつは静磁エネルギーが、電子同士のスピンが反平行の時より平行の方が低くなる。もうひとつは、
原子間の電子の移動はスピンが反平行の時のみ許されるというものである。原子$a,b$間の交換相互作用のハミルトニアン$\mathscr{H}_\text{ex}$は式\eqref{eq:交換相互作用}のようになる。\cite{strong1948}
\begin{equation}\label{eq:交換相互作用}
  \mathscr{H}_{\text{ex}} = -2J \symbfit{S}_a \cdot \symbfit{S}_b
\end{equation}
$J > 0$のとき原子$a,b$は平行である方がエネルギーが低く、$J < 0$のとき反平行である方がエネルギーが低くなる。超交換相互作用においては、$\ce{O^{2-}}$の$2p$軌道と磁性イオンの$3d$軌道が弱い共有結合を作り、
マイナスのスピンは磁性イオンの軌道に移動する。この仕組みにより$\ce{O^{2-}}$を介して、磁性イオンは反平行に配列することになる。

\subsection{単磁区構造}
磁性体は原子磁石に注目すると、磁壁を生成し磁区を形成する。このとき、磁区の大きさは静磁エネルギーと磁壁のエネルギーに左右される。
ここで静磁エネルギーとは、強磁性体自身が作る反磁場に対して、反対方向に作る磁化によるエネルギーである。静磁エネルギーは式\eqref{eq:静磁エネルギー}のように書ける\cite{strong1948}。
\begin{equation}\label{eq:静磁エネルギー}
 E_\text{d} = \frac{1}{2}\mu_0 D M^2
\end{equation}
ここで$D$は反磁場係数である。磁区が小さく互いに反平行を向くことで、静磁エネルギーは小さな値を取る。次に磁壁エネルギーとは磁区間の遷移層である。これは前述の通り、スピンの回転による
交換エネルギーによるものであり、磁壁が多くなるほど磁壁エネルギーが大きくなる。異方性定数$K_1$を持つ場合、単位面積あたりの磁壁のエネルギー$r$は、
係数$k$と交換スティフネス定数$A$を用いることで式\eqref{eq:磁壁エネルギー}のように表すことができる\cite{strong1948}。
\begin{equation}\label{eq:磁壁エネルギー}
  r = k \sqrt{AK_1}
\end{equation}
この二つのエネルギーの和が最小になるとき磁区の大きさが決まる。しかし単磁区構造は、磁壁を形成するよりもエネルギー的に安定である。
\begin{figure}[htbp]
  \centering
  \includegraphics[width=1.0\textwidth]{単磁区.png}
  \caption{磁区構造による静磁エネルギーと磁壁エネルギー}
  \label{fig:単磁区}
\end{figure}
このとき磁化過程は磁化の回転よる機構のみとなる。単磁区構造の臨界直径は$d_c$とかける。

球状単磁区粒子における磁化の回転機構について考える。外部磁場$B_0$を印加し、磁化容易軸と角度$\theta_0$を成すように印加した。磁化は磁化容易軸から$\theta$だけ傾いているとすると、磁場中での磁気モーメントのエネルギーは
式\eqref{磁気エネルギー}のように記述できる\cite{strong1948}。
\begin{equation}\label{磁気エネルギー}
  E(\theta) = K_{\text{u}}V\sin^2{\theta} - M_{\text{s}}VB_0\cos{(\theta - \theta_0)} + C
\end{equation}
ここで$K_{\text{u}}$は一軸磁気異方性定数、$V$は粒子の体積、$C$は定数である。このエネルギーを$\theta$で微分し、エネルギーを極小にする角度$\theta$を計算すると、
\begin{equation}
  \frac{dE(\theta)}{d\theta} = 2 K_u V \sin{\theta} \cos{\theta} + M_s V B_0 \sin{(\theta - \theta_0)} = 0
  \label{微分磁気エネルギー}
\end{equation}
式\eqref{微分磁気エネルギー}を満たす角度$\theta$とわかる。この現象は単磁区構造における、ヒステリシス現象の機構の一つとなっている。単磁区粒子の磁化容易軸はランダムな方向分布を持つ場合、全体の磁化曲線は磁化容易軸$\theta_0$のヒステリシス曲線の重ね合わせとして得られる。よって単磁区粒子においてもヒステリシス曲線が現れることになる。

\subsection{磁気緩和現象}
ナノ微粒子の磁気緩和現象では、Néel緩和とBrown緩和の2つが知られている。Néel緩和とは超常磁性体が熱エネルギーによって、粒子内で磁化がランダムに回転する機構である。
ここでの緩和時間は式\eqref{eq:ネール}で表される。
\begin{equation}\label{eq:ネール}
  \tau_\text{N} = \tau_0 \exp{\left(\frac{K V}{k_\textrm{B} T}\right)}
\end{equation}
ここで$\tau_N$はNéel緩和時間、$K$は磁気異方性定数、$V$は粒子の体積、$k_\textrm{B}$はボルツマン定数、$T$は絶対温度である。
Brown緩和は液体中に分散したナノ微粒子で起こり、緩和時間は式\eqref{eq:ブラウン}で表される。
\begin{equation}\label{eq:ブラウン}
  \tau_\text{B} = \frac{3 \eta V_h}{k_\textrm{B} T}
\end{equation}
ここで$\tau_\text{B}$はBrown緩和時間、$\eta$は流体粘度、$V_h$は粒子の流体力学的体積である。


\subsection{逆ヒステリシス現象}
逆ヒステリシスという現象は図\ref{fig:逆ヒステリシス}のような負の残留磁化を持つ特徴的なヒステリシスループとして数多く観察されている。
\begin{figure}[htbp]
  \centering
  \includegraphics[width=1.0\textwidth]{逆ヒステリシス.pdf}
  \caption{逆ヒステリシスループ}
  \label{fig:逆ヒステリシス}
\end{figure}
Yangらは、\ce{Co}ナノ微粒子を用いた粒子間相互作用による、逆ヒステリシスループについて報告している\cite{Yang2008}。試料作製では、レーザー照射を行うことで粒径の小さい超常磁性相を持つものと、
粒径の大きな強磁性相を持つものが混在する系を作製した。この試料での磁化過程を、4つの仮定とStoner-Wohlfarthモデルに基づいて説明する\cite{Stoner1948}。4つの仮定は、
\begin{enumerate}
  \renewcommand{\labelenumi}{(\roman{enumi})}
  \item 2つの異なる磁性相は、粒径が大きい強磁性の\ce{Co}ナノ微粒子(sample-LG)と粒径の小さい超常磁性相の\ce{Co}ナノ微粒子(sample-SM)に起因する。
  \item sample-LGの周囲に存在するsample-SMの総磁化($M_\text{SM}$)は単一のsample-LGの磁化($M_\text{LG}$)よりも大きい
  \item sample-LGの磁化には大きな磁気異方性を持つ
  \item sample-LGの粒子間距離は長いためsample-LG間に相互作用は働かない
\end{enumerate}
Stoner-Wohlfarthモデルに基づいて説明すると、系の総エネルギーは下式のようになる\cite{Stoner1948}。
\begin{align}
E ={}&
 - M_{\text{SM}} V_{\text{SM}} H \cos(\theta_\text{SM}-\theta_\mathit{H})
 - M_\text{LG} V_\text{LG} H \cos(\theta_\text{LG}-\theta_\mathit{H}) \notag\\
&+ K_\text{SM} V_\text{SM} \sin^2 \theta_\text{SM}
 + K_\text{LG} V_\text{LG} \sin^2 \theta_\text{LG} \notag\\
&- J_{\mathrm{eff}} M_\text{SM} M_\text{LG}
 \cos(\theta_\text{SM}-\theta_\text{LG})
\label{eq:total_energy}
\end{align}

ここで、$V_\text{SM}、V_\text{LG}$はsample-SMとsample-LGの体積、$K_\text{SM}、K_\text{LG}$はsample-SMとsample-LGの磁気異方性定数、$\theta_\text{SM}、\theta_\text{LG}$はsample-SMとsample-LGの磁化方向
$\theta_H$は外部磁場の方向と磁化容易軸との間の角度、$J_\text{eff}$は粒子間の有効交換相互作用定数である。このとき系のエネルギーが極小となるのは下記に示すように4つの条件である。
\begin{enumerate}
  \renewcommand{\labelenumi}{(\roman{enumi})}
  \item $\theta_\text{SM} = 0 ,\theta_\text{LG} = 0$
  \item $\theta_\text{SM} = 0 ,\theta_\text{LG} = \pi$
  \item $\theta_\text{SM} = \pi ,\theta_\text{LG} = 0$
  \item $\theta_\text{SM} = \pi ,\theta_\text{LG} = \pi$
\end{enumerate}
これらの条件を用いて逆ヒステリシスループを説明していく。まず磁化が飽和する十分大きな正の磁場を印加したとき、全ての\ce{Co}ナノ微粒子は磁場方向に磁化し、(i)の状態をとる。
次に磁化を弱めていく過程で系の総エネルギーを低くするため、sample-SMはsample-LGによる反磁場により磁場と反対方向に磁化し、(ⅲ)の状態をとる。このときの全体の総磁化は下式のようになる。
\begin{equation}\label{eq:負の残留磁化}
  |M_{LG}| - |M_{SM}| \le 0
\end{equation}
式\eqref{eq:負の残留磁化}より磁場方向と反対の磁化を持つsample-SMの方が、磁化が大きいため負の残留磁化を持つことがわかる。
次に十分大きな負の磁場を印加した際、全ての\ce{Co}ナノ微粒子は磁場方向に印加し、(ⅳ)の状態をとる。
次に磁化を弱めていく過程で系の総エネルギーを低くするため、sample-SMとsample-LGの反磁場により磁場と反対方向に磁化し、(ⅱ)の状態をとる。
この時正の磁場を弱めた時と同様の現象が起き負の残留磁化を示す。このような超常磁性相と強磁性相の粒子間相互作用により、反強磁性的双極子相互作用が粒子間に働くため、逆ヒステリシスループが生じるとわかる。

\subsection{磁気共鳴イメージング(MRI)}
\label{sec:MRI}
\subsubsection{MRIの概要}
Magnetic Resonance Imaging(MRI)は、核磁気共鳴(NMR)を利用している。主に水素の原子核の励起、緩和を利用している。\\
MRI装置の概要を図\ref{fig:MRI_overview}にしめす。磁石及び交流磁場コイルで測定対象に磁場を加え、対象の原子核を歳差運動させる。このとき歳差運動と同じ周波数である、Radio Frequency Pulse(RFパルス)を
照射することで励起される。緩和過程において、RFコイルに対象の原子核の磁気モーメントによる誘導電流が生じる。この電流を変換することで核磁気共鳴画像を得ることができる\cite{KitaokaMRI}。\\ 
\begin{figure}[htbp]
  \centering
  \includegraphics[width=0.8\textwidth]{MRI概略図.png}
  \caption{MRIの概要}
  \label{fig:MRI_overview}
\end{figure}
\subsubsection{核磁気共鳴}
電子または原子核のスピン$\symbfit{I}$に対応する磁気モーメントは以下の式になる\cite{KitaokaMRI}。
\begin{equation}
  \symbf{\mu} = \gamma \hbar \symbfit{I}
\end{equation}
ここで、$\mu$は磁気モーメント、$\gamma$は磁気回転比、$\symbfit{I}$はスピン角運動量、$\hbar$換算プランク定数である。\\
これに外部磁場$\symbfit{B_0}$が加わると、トルク$\symbfit{\mu} \times \symbfit{B_0}$を生じ、以下のような運動方程式に従って運動する。
\begin{equation}
  \hbar \frac{d\symbfit{I}}{dt} = \frac{1}{\gamma} \frac{d\symbf{\mu}}{dt} = \symbf{\mu} \times \symbfit{B_0}
\end{equation} 
磁場の方向を$z$軸方向にとると
\begin{equation}
  \frac{d\mu_x}{dt} = \gamma B_0 \mu_y, \quad \frac{d\mu_y}{dt} = - \gamma B_0 \mu_x, \quad \frac{d\mu_z}{dt} = 0
\end{equation}
となるので一般解は以下の式になる。
\begin{equation}
  \mu_x = A \cos (\omega_0 + \alpha), \mu_y = A \sin (\omega_0 + \alpha),\quad \mu_z = \text{const.}
\end{equation}
この式から、磁気モーメントは外部磁場の周りを角周波数$\omega_0 = \gamma B_0$で回転運動することがわかる。これをラーモアの歳差運動という\cite{KitaokaMRI}。

磁場中での磁気モーメントのエネルギーは
\begin{equation}
  U = - \symbf{\mu} \cdot \symbfit{B_0} = -\mu_z B_0 = - \gamma \hbar B_0 J_z
\end{equation}
となる。ここで$J_z$のとりうる値は$-m, -m+1, \ldots, m-1, m$であり、これをゼーマン準位と呼ぶ\cite{KitaokaMRI}。ここで$m$はスピン量子数と呼ぶ。隣り合う準位間のエネルギー差は
\begin{equation}
  |\Delta U| = |\gamma \hbar B_0| = |\hbar \omega_0|
\end{equation}
となっている。これはラーモア周波数に等しい振動数の電磁波のエネルギーがゼーマンエネルギー準位の差に等しいため、磁気モーメントがこの電磁波を吸収して高エネルギー準位に遷移する。この現象を核磁気共鳴(NMR)という。\\

電磁波を照射する前では、磁気モーメントは巨視的磁化$M$の成分は上向と下向きの磁気モーメントの個数の差により$z$軸正の向きに磁化が生じる。$x$-$y$平面内では磁気モーメントが様々な位相で歳差運動をしているため、$x$-$y$平面内の巨視的な磁化はゼロである。\\
核磁気共鳴(NMR)は、外部磁場中でスピンを持つ原子核が特定の周波数で共鳴吸収を示す現象です。\\
\begin{figure}[htbp]
  \centering
  \includegraphics[width=0.6\textwidth]{MRI.png}
  \caption{核磁気共鳴(NMR)とラーモアの歳差運動の概略図}
  \label{fig:MRI}
\end{figure}

励起電磁波による巨視的磁化の運動を考えるため、$z$軸を軸に$\omega_0$で回転する回転座標系$x'$-$y'$-$z'$を導入する。磁化$M$は磁気モーメントの$\mu$の集合なので、$\symbfit{M}$と磁化の角運動量$\symbfit{L_M}$は、
\begin{equation}
  \symbfit{M} = \gamma \symbfit{L_M}
\end{equation}
となる。励起電磁波による磁界成分を$\symbfit{B_1}$とすると、トルク$\symbfit{\mu} \times \symbfit{B_1}$により運動方程式は
\begin{equation}
  \frac{d\symbfit{M}}{dt} = \gamma [\symbfit{M} \times \symbfit{B_1}],\frac{d\symbfit{L_M}}{dt} = [\symbfit{M} \times \symbfit{B_1}]
\end{equation}
となる。ここで、$\symbfit{B_1}$は$x-y$平面内にある交流磁場であり、$\symbfit{B_1} = B_1(\cos \omega t \symbfit{i}$ + $\sin \omega t \symbfit{j})$($\symbfit{i}$は$x$軸の単位ベクトル、$\symbfit{j}$は$y$軸の単位ベクトル)で表される。\\
回転座標系での磁化の時間変化は以下の式で与えられる。
\begin{equation}  
  \left(\frac{d\symbfit{M}}{dt}\right)_{\text{rot}} = \gamma [\symbfit{M} \times \symbfit{B_1}] - [\symbfit{\omega_0} \times \symbfit{M}]
\end{equation}
ここで$\symbfit{M}=({M_x},{M_y},{M_z})$、$\symbfit{B_1}=(B_1,0,0)$、$T=0$で$\symbfit{M}=\symbfit{M_0}$で解くと、
\begin{equation}
  M_x' = 0,M_y' = M_0\cos(\gamma B_1t),M_z' = -M_0\sin(\gamma B_1t)
\end{equation}
となる。これはx軸は中心に$M$が倒れていくことを示している。固定座標系では$M$は$\omega_0$で回転しながら倒れる角度は
\begin{equation}
  \theta = \gamma B_1 t
\end{equation}
で与えられる。つまり$B_1$の大きさを一定にすれば$\theta$ は電磁波の照射時間を$t$で決まる。この$\theta$ をフリップ角という。フリップ角\qty{90}{\degree}の電磁波を\qty{90}{\degree}パルス、フリップ角\qty{180}{\degree}の電磁波を\qty{180}{\degree}パルスと呼ぶ\cite{KitaokaMRI}。

\subsubsection{磁気緩和}
励起電磁波照射後、フリップ角$\theta$まで励起された時の$\symbfit{M}$の成分は固定座標系で
\begin{equation}
  M_{x-y} = M_0 \sin \theta ,M_z = M_0 \cos \theta
\end{equation}
であり$(M_z,M_{x-y})=(M_0,0)$になるまでの過程を緩和という。それぞれの成分の時間成分は以下の式になる。
\begin{equation}
  M_{x-y} = M_0 \sin \theta \exp\left(-t/T_2\right), M_z = M_0 - (M_0 - M_0 \cos \theta)\exp\left(-t/T_1\right)
\end{equation}
ここで縦緩和時間$T_1$は、$z$軸成分$M_z$が$M_0$に戻るまでの時間を表し、横緩和時間$T_2$は$x$-$y$平面内の成分$M_{x-y}$が0になるまでの時間し、どちらも緩和時間と呼ばれる時定数である。
$T_1$緩和は$\beta$群(高エネルギー準位)から$\alpha$群(低エネルギー準位)への遷移、$T_2$緩和は歳差運動の位相の分散で説明される\cite{KitaokaMRI}。

\subsection{Spin Echo法によるMRI測定} %石川 宮野 神田 小池 三池 坂井
\subsubsection{パルスシーケンス}
MRIには前節で述べたような\qty{90}{\degree}パルス,\qty{180}{\degree}パルスを組み合わせ、どのパルスをいつ照射するかを表したパルスシーケンスが必要になる。

パルスシーケンスには、Spin Echo,Fast Spin Echoなど様々な種類があり、それぞれ得られるMR画像や測定に要する時間が変わる\cite{KitaokaMRI}。臨床においては、どのパルスシーケンスを使うかも重要である。\\
\subsubsection{Spin Echo法}
Spin Echo法はMRIにおける基本的なパルスシーケンスであり、\qty{90}{\degree},\qty{180}{\degree}パルスの2種類を用いられる。設定する変数はエコー時間(TE:Echo Time)とリピート時間(TR:Repetition Time)である。
TEは\qty{90}{\degree}パルス照射から信号を取得するまでの時間、TRは\qty{90}{\degree}パルスと次の\qty{90}{\degree}パルスの間隔である。ここでTEを$t_\text{TE}$、TRを$t_\text{TR}$と表す。
まず$t = 0$において\qty{90}{\degree}パルスを照射し、次に$t = \frac{t_\text{TE}}{2}$にて\qty{180}{\degree}パルス、最後に $t = t_\text{TR}$にて\qty{90}{\degree}パルスが照射される。
これがSpin Echo法の一回分の測定である。実際には信号強度を強めるためにこの一回分の測定を複数回繰り返し行い、MRシグナルを積算していく。
Spin Echo法のパルスシーケンスを図\ref{fig:SpinEcho}に示す。\\
\begin{figure}[htbp]
  \centering
  \includegraphics[width=0.9\textwidth]{MRI_spin-echo.png}
  \caption{Spin Echo法のパルスシーケンス}
  \label{fig:SpinEcho}
\end{figure}
\subsubsection{Spin Echo法によるT1, T2強調画像}
このようにして得られたSpin Echo法のMRシグナルではその強度が以下の式で表されることが知られている\cite{KitaokaMRI}。
\begin{equation}
  I_\text{SE} = c \cdot \rho \cdot \exp\left(-\frac{t_\text{TE}}{T_2}\right) \cdot \left(1 - \exp\left(-\frac{t_\text{TR}}{T_1}\right)\right)
\end{equation}\label{MRシグナル}
ここで$\rho$は断層面内のプロトン密度、$c$は測定状況に依存する環境定数である。

$t_\text{TE}=0$に設定するのは原理上不可能であるため、実際には$t_\text{TE} \ll t_\text{TR}$となる値を設定することで(24)式は以下のようになる。
\begin{equation}
  I_\text{SE} = c \cdot \rho \cdot \left(1 - \exp\left(-\frac{t_\text{TR}}{T_1}\right)\right)
\end{equation}
これが$T_1$緩和曲線の理論式である。臨床においても同じように$T_1$強調画像を得られる。また、この時、$T_1$の逆数を取った値を緩和率$R_1$と呼ぶ。

次に$t_\text{TR} = \infty(t_\text{TR} \gg t_\text{TE})$とした場合を考える。このとき(24)式は以下のようになる。
\begin{equation}
  I_\text{SE} = c \cdot \rho \cdot \exp\left(-\frac{t_\text{TE}}{T_2}\right)
\end{equation}
これが$T_2$緩和曲線の理論式であり$T_2$強調画像を得られる。また、この時$T_2$の逆数をとった値を緩和率$R_2$とよび、造影効果を表すパラメータである。

また$t_\text{TR} \gg T_1$、$t_\text{TE} \ll T_2$と設定すると、(24)式は以下のようになる。
\begin{equation}\label{eq:MRIシグナル}
  I_\text{SE} = c \cdot \rho 
\end{equation}
この式は$T_1$、$T_2$緩和の影響を排除した\ce{^1H}のMRシグナルであり、プロトン密度強調画像を得られる。

このようにSpin Echo法を用いることで様々な強調像を得られ、基本的なMRI撮像の基本的なシーケンスとして使われる\cite{KitaokaMRI}。

\clearpage

\subsection{磁気ナノ微粒子イメージング(MPI)}
\subsubsection{MPIの概要}
\begin{figure}{H}
  \centering
  \includegraphics[width=1.0\textwidth]{MPI.pdf}
  \caption{MPIの概要}
  \label{fig:MPI_overview}
\end{figure}
Magnetic Particle Imaging(MPI)は2005年にB. GleichとJ. Weizeneckerによって提案されたイメージング手法である\cite{Gleich2005}。
MPIは超常磁性ナノ微粒子の非線形的な磁気特性を利用したものであり、外部から交流磁場を印加すると、ナノ微粒子の磁化はLangevin関数に従って応答し、
高磁場強度において磁化が飽和するため、短形波のような磁化応答を示す\cite{Gleich2005}。この短形波をフーリエ変換し、MPIシグナルとして基本波の3倍の周波数応答である、
第三高調波応答のシグナルを用いる。

\section{実験}
\subsection{GdドープMn-Zn ferriteナノ微粒子の作製}
GdドープMn-Zn ferriteは下記の化学反応式より湿式混合法(特許第3933366号)を用いて作製した。
\begin{equation}\label{eq:湿式混合法}
  \begin{aligned}
 \ce{(0.8-$x$)MnCl2 + 0.2ZnCl2 + $x$GdCl3 + 2FeCl2 + 3Na2SiO3 + $x$NaOH}\\
 \ce{-> Mn_{0.8-$x$}Zn_{0.2}Gd_{$x$}Fe2O4 + 3SiO2 + {6+$x$}NaCl}
  \end{aligned}
\end{equation}

金属塩とアルカリを純水に溶かしたあと、混合し中和反応を起こし水酸化物沈殿を作製した。
その後、遠心分離を行い洗浄し、乾燥させた試料を粉砕し焼成した。
表\ref{tbl:reagents}に$\ce{Mn_{0.8-x}Zn_{0.2}Gd_{x}Fe2O4}$ナノ微粒子を合成するため以下の試薬をそのまま使用した。\\

\begin{table}[htbp]
\centering
\caption{GdドープMn--Zn ferriteナノ微粒子の作製に用いた試薬}
\label{tbl:reagents}
\begin{tabular}{lll}
\hline
試薬名 & 組成式 & 純度・製造元 \\
\hline
塩化マンガン四水和物 &
$\ce{MnCl2 . 4H2O}$ &
99.9\%, 富士フィルム和光純薬株式会社 \\

塩化亜鉛 &
$\ce{ZnCl2}$ &
98.0\%, 富士フィルム和光純薬株式会社 \\

塩化ガドリニウム六水和物 &
$\ce{GdCl3 . 6H2O}$ &
99.0\%, 富士フィルム和光純薬株式会社 \\

塩化鉄(II)六水和物 &
$\ce{FeCl2 . 6H2O}$ &
99.0\%, 富士フィルム和光純薬株式会社 \\

メタケイ酸ナトリウム九水和物 &
$\ce{Na2SiO3 . 9H2O}$ &
99.9\%, 富士フィルム和光純薬株式会社 \\

水酸化ナトリウム &
$\ce{NaOH}$ &
99.0\%, 富士フィルム和光純薬株式会社 \\
\hline
\end{tabular}
\end{table}

\ce{Gd}ドープMn-Zn ferriteナノ微粒子は、式\eqref{eq:湿式混合法}に示す組成式を目標に、$1/200$ mol--$1/10$ molのサンプルを作製した。
まず純水\qty{50}{mL}に塩化マンガン四水和物、塩化亜鉛、塩化ガドリニウム六水和物、塩化鉄(II)六水和物を溶解させ、金属イオン溶液を調整した。
次に純水\qty{100}{mL}にメタケイ酸ナトリウム九水和物、水酸化ナトリウムを溶解させ、アルカリ溶液を調整した。金属イオン溶液は\ce{Gd}のドープ量$x$に合わせて
秤量した。アルカリ溶液はメタケイ酸ナトリウム九水和物は目的のサンプル\qty{1}{mol}に対して\qty{3}{mol}のアモルファス$\ce{SiO2}$が生成されるように秤量した。
水酸化ナトリウムは塩基不足を補うため、塩化ガドリ二ウム$x$ molに対して$x$ mol秤量した。金属塩とアルカリを混合し、マグネティックスターラーと撹拌子で
\qty{350}{rpm}、室温で15分撹拌した。得られた沈殿物を(PP)製\qty{50}{mL}遠沈管にうつして遠心分離を\qty{3600}{rpm}で3分を2回、15分を1回行った。洗浄後、沈殿物を\qty{50}{\degreeCelsius}の乾燥炉で
2日間以上乾燥させた。この試料を乳鉢で15分間粉砕し、粉末状のものをアズプリ(焼成前駆体)とした。
このアズプリを焼成炉を用いて図\ref{fig:焼成プログラム}に示す焼成プログラムで焼成した。
\begin{figure}[htb]
  \centering
   \includegraphics[width=0.5\textwidth]{焼成プログラム.png}
  \caption{焼成プログラム}\label{fig:焼成プログラム}
\end{figure}
このとき\ce{Mn}、\ce{Gd}が酸化して不純物の生成を防ぐため、\qty{40}{mL/min}の\ce{Ar}を常に流しながら行った。焼成温度$Z$を変えることで粒径を調節できるため、
様々な粒径を\qtyrange{1050}{1200}{\kelvin}の間で様々に変化させることで行った。焼成前のサンプル、焼成後のサンプルは図\ref{fig:sample}に示す。
\begin{figure}[htb]
  \centering
   \includegraphics[width=0.8\textwidth]{サンプルGd.pdf}
  \caption{(a)$\ce{Mn_{0.8-x}Zn_{0.2}Gd_{x}Fe2O4}$のアズプリの写真(b)$\ce{Mn_{0.8-x}Zn_{0.2}Gd_{x}Fe2O4}$の焼成後の写真}\label{fig:sample}
\end{figure}

\subsubsection{粉末X線回折測定(XRD)}
作製した磁気ナノ微粒子の結晶構造を調べるため、粉末X線回折測定はリガク社製のMiniFlex IIとRINT2500を用いて行った。各装置は図\ref{fig:XRD}である。
\begin{figure}[htbp]
  \centering
   \includegraphics[width=0.9\textwidth]{XRD装置.pdf}
  \caption{(a)RINT2500の写真(b)MiniFlex IIの写真}\label{fig:XRD}
\end{figure}
MiniFlex IIでは、Cu-K$\alpha$ 線(波長\qty{1.5406}{Å})を用い、管電圧\qty{30}{kV}、管電流\qty{15}{mA}、測定範囲\qtyrange{10}{80}{\degree}、ステップ幅\qty{0.15}{\degree}、測定速度\qty{2}{\degree/min}の条件で連続測定を行った。
RINT2500では、Cu-K$\alpha$ 線(波長\qty{1.5406}{\angstrom})を用い、管電圧\qty{40}{kV}、管電流\qty{45}{mA}で測定を行った。1つ目の測定は$2\theta = $\qtyrange{10}{80}{\degree}、ステップ幅\qty{0.15}{\degree}、測定速度$\qty{2}{\degree/min}$
の条件で連続測定を行った。二つ目の測定では、スピネル構造のミラー指数(311)(440)のピークの高精度測定を目的として、$2\theta = $\qtyrange{33}{37}{\degree}、\qtyrange{60}{64}{\degree}、ステップ幅\qty{0.004}{\degree}、測定方法は
一定時間内のカウント数を計測するFixed Time(FT)法で行った\cite{Warren1950}。
試料ホルダーは、株式会社リガク製のガラス試料ホルダー(\qtyproduct{20x20x0.5}{mm})を用いた。

XRDパターンの解析は、株式会社リガク製のPDXL2を用いて行った。XRDパターンのデータベースは国際結晶データセンター(ICDD)のPDF-4+を用いた。
また、粒径に関してはFP法によって算出された結晶子サイズとして評価を行った\cite{Cheary1992}。

\subsubsection{蛍光X線分析(XRF)}
作製した磁気ナノ微粒子の各金属の組成比を調べるために、蛍光X線分析(XRF)測定を行った。XRF測定は、横浜国立大学の機器分析センターにある日本電子株式会社製の、
JSX-3100RⅡを用いて行った。測定は専用のカップに\qty{10}{mg}程度入れ測定を行った。\ce{Si}元素を含めた測定では、専用の多孔質フィルムをし密閉することで、真空引きを行った。
測定条件はRhのX線管球を用い、管電圧\qty{30}{kV}、管電流\qty{1}{mA}、最適化係数値\qty{25000}{cps}、測定条件\qty{100}{s}の条件で測定を行った。

\subsubsection{X線吸収微細構造解析(XAFS)}
作製した磁気ナノ微粒子の局所構造解析を行うため、高エネルギー加速器研究機構(KEK)において、フォトンファクトリーのBL-9Cのビームラインを用いてXAFS測定を行った。
作製したサンプルを適した形状に成型するため窒化ホウ素($\ce{BN}$)を混合し、油圧プレス機を用いてペレット状にし、適切なX線吸収強度を示すよう加工した。
XAFS測定には、\ce{Mn}、\ce{Zn}、\ce{Fe}のK吸収端と\ce{Gd}の$L_2$吸収端で測定を行った。XAFS測定によって得たデータの解析は、XAFS解析ソフトウェアAthena
を用いて行った\cite{Ravel2005}。

\subsection{磁化測定}
作製した磁気ナノ微粒子の磁気特性を調べるために超伝導量子干渉装置(SQUID)を用いて行った。
測定装置は大阪大学 大学院理学研究科 附属 熱 · エントロピー科学研究センターにある Quantum Design 社製 MPMS-1S、
東京科学大学 フロンティア材料研究所 川路研究室にある Quantum Design 社製 MPMS-7(AC 測定オプション付き)、
横浜市立大学 理学部 理学科 生命ナノシステム科学研究科 物質システム科学専攻 山田研究室にある Quantum Design 社製 MPMS-XL を用いた。
サンプルの測定を行う際に\qty{100}{Oe}の磁場を印加してセンタリングを行い、全ての測定の前に消磁処理(デガウス)を行った。$T = \qty{300}{\kelvin}$では\qty{100}{Oe}、
それ以下の温度ではサンプルの保磁力が大きくなるため、\qty{1000}{Oe}の磁場強度からデガウスを行った。

SQUID磁束系での磁化測定のため図のようなストローと呼ばれるものを作製した。ゼラチンカプセルに測定サンプルを入れ、その後脱脂綿を詰めることで
サンプルを固定した。そのゼラチンカプセルをストローの中にいれ、さらにカプトンテープを用いてカプセルが動かないように固定した。
最後に全体にピンセットを用いて圧力平衡用の穴を30か所ほど開けた。ゼラチンカプセルに封入したサンプルの質量はXRFでの計測の結果を用いて
、式\eqref{eq:純サンプル質量}により$\ce{SiO2}$を抜いた純サンプル量を計算した。これにより$\ce{Mn_{0.8-x}Zn_{0.2}Gd_{x}Fe2O4}$の質量を計算することができる。
計算の結果は表\ref{tbl:SampleList}に示す。
\begin{figure}[htb]
  \centering
   \includegraphics[width=0.7\textwidth]{ストロー.jpg}
  \caption{SQUID測定のため作成したストローの写真}\label{fig:ストロー}
\end{figure}

\begin{equation}
m_\textrm{pure} = m_\textrm{sample} \frac{X}{X + (A_{\ce{Si}}+2A_{\ce{O}})(3 x_{\ce{Si}}/(x_{\ce{Mn}} + x_{\ce{Zn}} + x_{\ce{Gd}} + x_{\ce{Fe}}))}
\label{eq:純サンプル質量}
\end{equation}
\begin{equation}
X = A_{\ce{Mn}}x_{\ce{Mn}} + A_{\ce{Zn}}x_{\ce{Zn}} + A_{\ce{Gd}}x_{\ce{Gd}} + A_{\ce{Fe}}x_{\ce{Fe}} + 4A_{\ce{O}}
\label{eq純サンプル質量補足}
\end{equation}
ここで$m_\textrm{pure}$は純サンプル質量、$m_\textrm{sample}$は実際のサンプル質量、$A_E$は元素$E$の原子量、$x_E$はXRFで測定した元素$E$のモルパーセントである。
\begin{table}
  \caption{\textbf{本研究で作製したGdドープ量別サンプルの一覧}}\label{tbl:SampleList}
  \centering
  \begin{tabular}{ccc}
    \hline
    サンプル名 & サンプル質量(mg) & 純サンプル質量(mg)\\
    \hline
    Gd-0(\qty{14.6}{nm}) & 21.80 & 16.73\\
    Gd-1(\qty{14.2}{nm}) & 21.38 & 15.82\\
    Gd-2(\qty{14.5}{nm}) & 22.00 & 16.30\\
    Gd-3(\qty{14.8}{nm}) & 22.40 & 17.70\\
    Gd-4(\qty{14.9}{nm}) & 21.60 & 16.00\\
    Gd-5(\qty{14.7}{nm}) & 20.15 & 15.40\\
    Gd-6(\qty{10.5}{nm}) & 19.30 & 14.80\\
    Gd-6(\qty{14.1}{nm}) & 21.81 & 16.70\\
    Gd-6(\qty{16.9}{nm}) & 19.60 & 15.00\\
    Gd-6(\qty{23.5}{nm}) & 21.20 & 16.30\\
    Gd-7(\qty{14.1}{nm}) & 22.10 & 16.90\\
    Gd-8(\qty{14.4}{nm}) & 21.00 & 16.30\\
    Gd-9 (\qty{15.3}{nm}) & 50.00 & 37.10\\
    Gd-20(\qty{13.3}{nm}) & 19.50 & 14.86\\
    \hline
  \end{tabular}
\end{table}

磁化曲線の測定はそれぞれの磁場強度での磁化を測定した。磁場は\qtyrange{0}{100}{Oe}においては初透磁率を見るため\qty{25}{Oe}間隔で印加し、
そこから300,1000,3000,\qty{5000}{Oe}間隔で\qty{1}{T}まで印加した。この間隔を基準にして$\pm \qty{1}{T}$の範囲で測定した。
温度に関しては\qty{5}{\kelvin},\qty{200}{\kelvin},\qty{300}{\kelvin}で測定を行った。

交流磁化の温度依存性(AC-T)測定では温度を変化させ、各温度で周波数\qty{10}{Hz},\qty{100}{Hz},\qty{500}{Hz},\qty{1000}{Hz}で測定した。
温度は\qtyrange{150}{300}{\kelvin}の間で\qty{10}{\kelvin}の間隔で変化させ、\qty{1}{Oe}の磁場強度で印加した。

残留磁化の温度依存性($M_{\text{r}}$-$T$)測定は$M-H$曲線における残留磁化$M_{\text{r}}$のみを測定した。
\qty{1}{T}の磁場をかけて磁化が飽和したあと、\qty{0}{Oe}にした際の磁化を測定した。温度は\qtyrange{300}{80}{\kelvin}まで\qty{20}{\kelvin}間隔、
\qtyrange{80}{5}{\kelvin}まで\qty{18.7}{\kelvin}間隔で測定を行った。

\subsection{磁気ナノ微粒子イメージング(MPI)}
\begin{figure}[H]
  \centering
   \includegraphics[width=1.0\textwidth]{MPI装置1.pdf}
  \caption{MPIシグナルの測定に用いたコイル。(a)交流磁場発生コイル。(b)自作した差動巻きのピックアップコイル。}\label{fig:MPI2}
\end{figure}
\begin{figure}[H]
  \centering
   \includegraphics[width=0.4\textwidth]{MPI装置1.png}
  \caption{冷却ピックアップコイルを用いたMPIシグナルの測定様子}\label{fig:MPI1}
\end{figure}
MPIシグナルとされる第三高調波測定は図に示すような自作の装置を用いて測定を行った。測定装置は、交流磁場発生用コイルとピックアップコイルの2つを、
さらにコイルの熱雑音によるノイズを低減するため\qty{77}{\kelvin}の液体窒素で冷却しながら測定を行った。交流磁場は、図に示す交流磁場発生用コイルに、
エヌエフ回路設計ブロック株式会社製BP4610のバイポーラ電源を用いて発生させた交流電源を流すことで印加した。ここで、サンプルに印加される磁場強度は
バイポーラ電源から直流電流を印加し、その電流における磁場強度を電子磁気工業社製のガウスメーターを用いて測定した。\qty{1000}{Hz}以下の低周波数帯では、
バイポーラ電源の示す電流値と実際に印加されている電流値に誤差がなかった。よって直流電源を流した時の磁場強度をもとに、目的の振幅に対応する
交流電源を印加した。液体窒素によるサンプルの温度低下は、断熱性に優れた硬質ウレタンフォーム製のカバーでサンプルを覆いながら測定を行った。

\subsection{磁気共鳴イメージング(MRI)}
MRI測定は、東京大学 関野研究室の協力のもと、BioSpec製 70/20USRを用いて測定を行った。MRI測定では、Gd-6の粒径別(\qty{10.5}{nm},\qty{14.1}{nm},\qty{16.9}{nm},\qty{23.5}{nm})
、そして寒天のSpin Echo法による$T_1,T_2$緩和測定を行った。
\subsubsection{MRI評価用ファントムの作製}

\textit{in vitro}でのMRIの緩和率の評価は、造影剤を水に分散させた状態のファントムと呼ばれるものを作ることで行う。ここで磁気ナノ微粒子の凝集を防ぐため、
超音波ホモジナイザーを用いて測定した。PP製の\qty{50}{mL}遠沈管に純水\qty{30}{mL}入れ、金属イオンの濃度が純水に対して\qty{1.0}{mM}となるように秤量した。
純水と測定サンプルを混ぜ超音波ホモジナイザーで20分処理し、純水に懸濁させた。寒天はビーカーに\qty{50}{mL}遠沈管に入れる純水に対して\qty{0.8}{wt\%}で秤量し、マグネティックスタラーを用いて
撹拌しながら、寒天の溶解温度である\qty{90}{\degreeCelsius}まで加温し溶かした。溶かした寒天は遠沈管に入れ、そのまま常温で放置して固化させた。
作製したファントムを図 \ref{fig:ファントム}に示す。
\begin{figure}[htb]
  \centering
   \includegraphics[width=0.4\textwidth]{ファントム.jpg}
  \caption{作成したファントムの写真}\label{fig:ファントム}
\end{figure}

\subsubsection{$T_1$,$T_2$緩和測定}
作製した磁気ナノ微粒子をファントムにしたものの、$T_1$,$T_2$緩和時間を評価した。ファントムはRFコイル内にテープで固定し、MRI装置の撮像部に静置した。
$T_1$緩和測定ではパルス系列のエコー時間(Echo Time)は\qty{2500}{ms}に固定し、繰り返し時間(Reception Time)を\qty{12}{ms}から\qty{48}{ms}間隔で測定を行った。
$T_2$緩和測定ではパルス系列の繰り返し時間は$TR$は\qty{11}{ms}から\qty{11}{ms}間隔に、エコー時間$TE$は\qty{12}{ms}に固定した。
これらにより$T_1$,$T_2$緩和曲線が得られ、緩和率$R_1$,$R_2$を算出した。

\subsection{アミノ基修飾}
アミノ基修飾は(3-アミノプロピル)トリエトキシシラン(APTES)を用いて作製した磁気ナノ微粒子の表面にアミノ基を導入した。ここで作製したサンプルは\ce{SiO2}包含されているが、アミノ基修飾させる上で十分量ではないため、
シリカ前駆体としてテトラエトキシシラン(TEOS)を用いてゾルゲル法であるst\"{o}ber法を応用して、サンプルの表面にシリカ層を形成した。
アミノ基修飾に使用したサンプルはGd-6を用いた。PP製\qty{50}{mL}遠沈管にEtOHを入れ、サンプルを\qty{70}{mg}を混ぜ、ホモジナイザーで30分間超音波処理を行い、
均一に分散させた。PP製\qty{500}{mL}三角フラスコに超音波処理したサンプルとエタノール(EtOH)を\qty{200}{mL}になるように入れた。さらに、\qty{0.0355}{mL} TEOS、\qty{10}{mL} APTES、
触媒として\qty{2}{mL} アンモニア水加えた。混合した溶液はメカニックスターラーで\qty{450}{rpm}、温度を\qty{85}{\degreeCelsius}にし24時間反応させた。
反応後、遠心分離を用いて\qty{3500}{rpm}、5分間を4回行い洗浄し、ニンヒドリン試験を行い遊離$-\ce{NH2}$基の存在がないことを確認した。
得られた沈殿物は\qty{55}{\degreeCelsius}の乾燥炉に入れ2日以上乾燥させた。
\begin{figure}[htbp]
  \centering
  \includegraphics[width=0.7\textwidth]{ニンヒドリン反応_遠心分離.png}
  \caption{遠心分離後のニンヒドリン反応}
  \label{fig:ニンヒドリン反応遠心分離}
\end{figure}

\section{結果及び考察}
\subsection{\ce{Gd}ドープMn-Zn ferriteの作製}
\ce{Gd}ドープMn-Zn ferriteは組成式$\ce{Mn_{0.8-x}Zn_{0.2}Gd_xFe2O4}$を目標に湿式混合法により作製した。
本研究で作製したサンプルを表\ref{tbl:Sample}に示す。
\begin{table}
  \caption{\textbf{本研究で作製したGdドープ量別サンプルの一覧}}\label{tbl:Sample}
  \centering
  \begin{tabular}{l|ccccccccccc}
    \hline
    サンプル名 & Gd-0 & Gd-1 & Gd-2 & Gd-3 & Gd-4 & Gd-5 & Gd-6 & 
    Gd-7 & Gd-8 & Gd-9 & Gd-20 \\
    \hline
    Gd含有量$x$ & 0 & 0.01 & 0.02 & 0.03 & 0.04 & 0.05 & 0.06 & 0.07 & 0.08 & 0.09 & 0.2 \\
    \hline
    粒径(nm) & 14.2 & 14.3 & 14.5 & 14.3 & 14.4 & 14.3 & 14.5 & 14.4 & 14.3 & 14.2 & 13.0 \\
    \hline
  \end{tabular}
\end{table}

\subsection{蛍光X線分析(XRF)}
本研究で作製した$\ce{Gd}$ドープMn-Zn ferriteナノ微粒子のXRF分析を行い得られたピークからFP法を行い各元素のモル比率を定量化した。
表\ref{tbl:XRF}は得られた各サンプルのモル比率である。\\
\begin{table}
  \caption{\textbf{蛍光X線分析で測定したGdドープMn-Zn ferriteナノ微粒子のモル\%}}\label{tbl:XRF}
  \centering
  \begin{tabular}{l|ccccc}
    \hline
    サンプル名 & Mn & Zn & Gd & Fe & Si\\
    \hline
    Gd-0 & 19.47(3) & 4.47(7) & 0 & 47.96(3) & 28.10(20)\\
    Gd-1 & 18.88(4) & 4.00(8) & 0.28(12) & 45.60(38) & 31.24(21) \\
    Gd-2 & 18.55(4) & 4.03(7) & 0.46(11) & 45.50(4) & 31.46(20)\\
    Gd-3 & 19.10(6) & 5.03(11) & 0.62(16) & 49.64(5) & 25.60(29)\\
    Gd-4 & 17.89(4) & 4.04(8) & 0.95(12) & 45.43(4) & 31.68(21)\\
    Gd-5 & 18.20(4) & 4.00(7) & 1.05(10) & 48.14(3) & 28.61(18)\\
    Gd-6 & 18.08(4) & 4.37(7) & 1.27(10) & 47.58(3) & 28.70(18)\\
    Gd-7 & 17.71(4) & 4.11(7) & 1.59(11) & 47.73(3) & 28.86(19)\\
    Gd-8 & 17.38(4) & 4.56(7) & 1.81(11) & 48.48(3) & 27.76(19)\\
    Gd-9 & 18.04(4) & 4.25(7) & 2.04(11) & 50.34(3) & 25.33(20)\\
    Gd-20 & 14.16(4) & 3.60(7) & 4.53(11) & 41.88(3) & 35.83(19)\\
    \hline
  \end{tabular}
\end{table}

\newpage

また表のモル比率から作製したサンプルの組成比を調べるために下式を使って計算した。\\
\begin{equation}
  X_M=\frac{3x_M}{x_{\ce{Mn}}+x_{\ce{Zn}}+x_{\ce{Gd}}+x_{\ce{Fe}}}
\end{equation}
ここで$X_M$は求めたい組成比、$x_M$は求めたい原子のモル比率、$x_{\ce{Mn}},x_{\ce{Zn}},x_{\ce{Gd}},x_{\ce{Fe}}$は各金属原子のモル比率である。
計算の結果は表\ref{tbl:XRF1}のようになった。

作製したサンプルは概ね秤量値通りの目的の組成比で作製することができた。

\begin{table}
  \caption{\textbf{蛍光X線分析で測定したGdドープMn-Zn ferriteナノ微粒子の組成}}\label{tbl:XRF1}
  \centering
  \begin{tabular}{l|ccccc}
    \hline
    サンプル名 & Mn & Zn & Gd & Fe & Si\\
    \hline
    Gd-0 & 0.812 & 0.187 & 0 & 2.001 & 1.172\\
    Gd-1 & 0.824 & 0.175 & 0.012 & 1.990 & 1.363 \\
    Gd-2 & 0.812 & 0.176 & 0.020 & 1.992 & 1.377\\
    Gd-3 & 0.770 & 0.203 & 0.025 & 2.002 & 1.032\\
    Gd-4 & 0.786 & 0.177 & 0.042 & 1.995 & 1.391\\
    Gd-5 & 0.765 & 0.168 & 0.044 & 2.023 & 1.202\\
    Gd-6 & 0.761 & 0.184 & 0.053 & 2.002 & 1.208\\
    Gd-7 & 0.747 & 0.173 & 0.067 & 2.013 & 1.217\\
    Gd-8 & 0.722 & 0.189 & 0.075 & 2.014 & 1.153\\
    Gd-9 & 0.725 & 0.171 & 0.082 & 2.022 & 1.018\\
    Gd-20 & 0.646 & 0.176 & 0.206 & 1.972 & 1.319\\
    \hline
  \end{tabular}
\end{table}

\subsection{粉末X線回折(XRD)}
\subsubsection{Gdドープ量別Mn-Zn ferriteのXRD}

$\ce{Gd}$ドープ量の異なる各サンプルのXRDパターンを、図\ref{fig:XRD}に示した。粒径は\qtyrange{13}{15}{nm}に揃えた。
全てのサンプルにおいて、XRD測定は連続法で測定を行った。
全てのサンプルにおいて、スピネル構造に対応するピークが確認されており、Fd-3mの空間群で指数付けすることができた。
また、$\theta = $\qtyrange{15}{30}{\degree}付近において、アモルファス$\ce{SiO2}$由来のブロードなピークが確認され、作製時にアモルファス$\ce{SiO2}$
が合成されていることがわかった。粒径の算出は、最も強いピーク強度を持つ(311)面において、FP法を適用することで、求めた値を採用している。\\
\begin{figure}[H]
  \centering
  \includegraphics[width=0.9\textwidth]{XRD_Gd0-20.pdf}
  \caption{Gdドープ量が異なる各サンプルのXRDパターン}
  \label{fig:XRD}
\end{figure}

Gd-20に関しては、図\ref{fig:XRDGd-20}からスピネル構造には見られないピークが観察されており、これは$\ce{Mn2Gd8(SiO4)6O2}$に対応するものと考えられる。六方晶構造に対応するピークが確認され、空間群はP$6_3/$mで指数付けをすることができた。
一般に希土類イオン(\ce{RE^{3+}})のスピネルフェライトへのドープでは、\ce{Fe^{3+}}と比較してイオン半径が大きいためスピネル格子への固溶限界が低いことが知られている。この固溶限界を超えると、斜方晶のferriteなどの二次相が生成することが報告されている\cite{Sharma2021}。
本研究におけるGd-20の試料作製では、組成比$\ce{Mn_{0.8-x}Zn_{0.2}Gd_xFe2O4}$を揃えるため、\ce{Gd}の秤量値を補正のため13\%ほど増やし調整している。さらに\ce{Gd}はイオン半径が大きくスピネル構造に入りにくいと考えられ、過剰に存在した\ce{Gd}がスピネル相として固溶せず、焼成過程において共存する$\ce{SiO2}$と反応した結果、$\ce{Mn2Gd8(SiO4)6O2}$が二次相として生成したものと考えられる。\\
\begin{figure}[H]
  \includegraphics[width=0.9\textwidth]{XRD_Gd-20.png}
  \caption{Gd-20のXRDパターン}
  \label{fig:XRDGd-20}
\end{figure}

また\ce{Gd}ドープによるピーク位置のシフトについて調べた。Braggの式から格子面間隔$d$が大きくなると、回折角度$\theta$は小さくなる\cite{Bragg1913}。
\begin{equation}
  2d \sin \theta = n \lambda
\end{equation}
ここで$d$は格子面間隔、$\theta$は回折角度、$n$は回折次数、$\lambda$はX線の波長である。今回作製したサンプルの各金属元素のイオン半径は以下のようになる。
表\ref{tbl:イオン半径}に示すように、$\ce{Gd}^{3+}$のイオン半径は\ce{Mn}、\ce{Zn}、\ce{Fe}の金属イオンより大きいため、
\ce{Gd}ドープ量が増加するにつれて低角側にシフトすると考えられる\cite{Shannon1976}。

\begin{table}[H]
  \centering
  \caption{\ce{Mn},\ce{Zn},\ce{Gd},\ce{Fe}のイオン半径(HS:高スピン、LS:低スピン)}\label{tbl:イオン半径}
   \begin{tabular}{lcccc}
    \hline
    元素 & 元素名 & 価数 (スピン状態) & 配位数 & 有効イオン半径(\AA) \\ \hline
     Mn & マンガン & 2+ & 4 & 0.66 \\
     Mn & マンガン & 2+ & 6 & 0.830 \\
     Zn & 亜鉛 & 2+  & 4 & 0.60 \\
     Gd & ガドリニウム & 3+  & 6 & 0.938 \\
     Fe & 鉄 & 3+ (HS) & 4 & 0.49 \\
     Fe & 鉄 & 3+ (LS) & 6 & 0.55 \\
     Fe & 鉄 &  3+ (HS) & 6 & 0.645 \\ \hline
  \end{tabular}
\end{table}
そこで最も回折強度の大きな(311)面におけるピーク位置と、
広角側の比較的大きな(440)面のピークを角度分解能を上げて測定をした。測定条件は角度分解能を上げ、FixedTime(FT)法で測定した。
\begin{figure}[htbp]
  \centering
  \includegraphics[width=1.0\textwidth]{XRDfitting.pdf}
  \caption{角度分解能を上げた$\ce{Gd}$ドープ量別でのXRDパターン(a)測定範囲$2\theta =$ \qtyrange{33}{37}{\degree}の(311)面のピーク(b)$2\theta =$ \qtyrange{60}{64}{\degree}の(440)面のピーク}
  \label{fig:XRD_peak}
\end{figure}
図\ref{fig:XRD_peak}が測定結果である。測定したデータはノイズが大きく、ピークの位置を判断することが困難である。そこで
測定データを、擬フォークト関数(ガウス分布関数とローレンツ分布関数の畳み込み)を使いピーク角度を検出した\cite{Balzar1993}。
擬フォークト関数は以下の式\eqref{eq:voigt}で表される。フィッティングした結果が図\ref{fig:XRDfitting}である。
\begin{equation}\label{eq:voigt}
y = y_0 + A \left[
m_u \frac{2}{\pi} \frac{w}{4\left(x - x_c\right)^2 + w^2}
+ \left(1 - m_u\right)
\frac{\sqrt{4\ln 2}}{\sqrt{\pi}\, w}
\exp\!\left(
-\frac{4\ln 2}{w^2}\left(x - x_c\right)^2
\right)
\right]
\end{equation}
ここで$y_0$はオフセット、$x_c$は中心位置、$A$は面積、$w$はガウス分布関数とフォークト関数のピーク強度の半分の高さにおける幅(FWHM)、$m_\text{u}$はプロファイル形状係数である。
\begin{figure}[H]
  \centering
  \includegraphics[width=1.0\textwidth]{fitting.pdf}
  \caption{図\ref{fig:XRD_peak}を擬フォークト関数によりフィッティングしたXRDパターン(a)(311)面のピーク(b)(440)面のピーク}
  \label{fig:XRDfitting}
\end{figure}
さらに$\ce{Gd}$ドープ量別の(311)面と(440)面のピーク角度から格子定数を算出した結果が図\ref{fig:lattice}である。
\begin{figure}[htbp]
  \centering
  \includegraphics[width=1.0\textwidth]{lattice.pdf}
  \caption{図\ref{fig:XRDfitting}から格子定数を算出しグラフ化したもの(a)(311)面のピーク(b)(440)面のピーク}
  \label{fig:lattice}
\end{figure}
これより$\ce{Gd}$-0から$\ce{Gd}$-8までピークが低角側にシフトしていることが確認でき、
Mn-Zn ferriteにイオン半径の大きい$\ce{Gd}$がドープされ、格子定数が増加したと説明できる。当研究室の先行研究である\ce{Gd}ドープMn ferriteと同じ挙動を示した\cite{Sakamoto2024}。
$\ce{Gd}$-9と$\ce{Gd}$-20に関してはピーク位置のシフトが広角側にシフトしている。これは$\ce{Gd}$過剰ドープにより、前述したように$\ce{Mn2Gd8(SiO4)6O2}$の六方晶構造ができ、$\ce{Gd}$が構造に入らず、
目的の試料は$\ce{Mn_{0.8-x}Zn_{0.2}Gd_{x}Fe2O4}$であるため、\ce{Gd}のドープ量が増加することで、$\ce{Mn}$の割合が減少し、相対的に$\ce{Zn}$の割合が増加する。$\ce{Zn}$イオンは他の金属イオンよりイオン半径が小さいため
格子定数が小さくなり、ピーク位置が広角側にシフトしたと考えられる。またPDXL2を用いて、ピーク角度の算出を行い格子定数を算出した結果は、図\ref{fig:lattice_PDXL}のようになり同じ傾向を示した。\\
\begin{figure}[H]
  \centering
  \includegraphics[width=0.7\textwidth]{lattice_PDXL.png}
  \caption{図\ref{fig:XRD_peak}をPDXL2より算出した格子定数}
  \label{fig:lattice_PDXL}
\end{figure}
\subsubsection{粒径別GdドープMn-Zn ferriteのXRD}
作製したGd-6のサンプルを、焼成温度を調整することで粒径\qtyrange{7.5}{23.5}{nm}の範囲で、6つのサンプルを作製した。これらの粒径別の各サンプルを\ce{Gd}ドープ量別で行ったXRDの測定条件と同様に測定を行った。全ての粒径別のサンプルにおいて図\ref{fig:Gd-6_size}に示した結果から、\ce{SiO2}のブロードなピークが見られ、スピネル構造を確認した。不純物相のピークは見られなかった。\\
\begin{figure}[H]
  \centering
  \includegraphics[width=0.6\textwidth]{Gd-6_粒径.png}
  \caption{Gd-6粒径別の各サンプルのXRDパターン}
  \label{fig:Gd-6_size}
\end{figure}

\newpage

\subsection{X線吸収微細構造測定(XAFS)}
\begin{figure}[H]
  \centering
  \includegraphics[width=0.8\textwidth]{XAFS.png}
  \caption{XAFSスペクトル}
  \label{fig:XAFS}
\end{figure}
作製した磁気ナノ微粒子の価数を調べるため、X線吸収微細構造(XAFS)測定を用いて構造解析を行った。作製した$\ce{Gd}$ドープ量別のサンプルを
\ce{Mn}、\ce{Zn}、\ce{Fe}のK吸収端において測定した。XAFSスペクトルは、吸収端付近のX-ray absorption near edge structure(XANES)スペクトルが得られた。
XANESスペクトルからは、電子状態などの中心原子の状態に依存する情報を得ることができる。そこで吸収端エネルギーのシフトを見ることで、各金属元素の価数を調べ、さらにpre-edge peakから結晶の対称性を調べた。
XAFSスペクトルの解析は、XAFS解析ソフトウェア(Athena)を用いて解析を行った。測定データのスペクトルをpre-edge lineとpost-edge lineを、それぞれ1次関数と3次関数を使って差し引いた。そこからpre-edge lineが0、EXAFS領域が1を中心に振動するように規格化を行った。

ここで、EXAFS領域については、今回作製したサンプルが金属元素の種類が多く、解析が複雑になってしまう。EXAFS領域の解析にあたりフィッティングを行うが、その結果の信頼性を次の式から定量的に調べる方法が知られている\cite{NihonXAFS2025}。
\begin{equation}\label{eq:Nyquist}
N_\text{ind} = \frac{2\Delta k \Delta R}{\pi} + m
\end{equation}
ここで$N_\text{ind}$は独立点の数、$\Delta k$はフーリエ変換の範囲、$\Delta R$は逆フーリエ変換の範囲、mは0,1,2のいずれかである。作製したサンプルはスピネル構造のため、最近接原子の全ては酸素原子であるため、第2配位圏の金属原子を調べる必要がある。ここで$\Delta k = 12$、$\Delta R = 2$とすると$N_\text{ind}$は16とし、中心元素
としてA-siteに入っている\ce{Zn}のパラメータ数を計算してみる。散乱経路数はA-siteの金属原子に\ce{Mn}、\ce{Zn}、\ce{Gd}、\ce{Fe}の4種類、B-siteの金属原子に\ce{Mn}、\ce{Gd}、\ce{Fe}の3種類で散乱経路数は7である。さらに各散乱経路に対して、最低でも3つのパラメータが必要であるため、パラメータ数は21となる。これより$N_\text{ind}$よりパラメータ数が多くなってしまうため、EXAFS領域の解析は行わない。

図\ref{fig:Fe-K}は、作製したサンプルの$\ce{Gd}$ドープ量別のXANESスペクトルを示し、Fe K吸収端エネルギーのシフトが見られないため、
$\ce{Gd}$ドープによる$\ce{Fe}$の価数の影響はないとわかる。\\
\begin{figure}[H]
  \centering
  \includegraphics[width=0.8\textwidth]{Fe-K_Gd.pdf}
  \caption{Fe K吸収端でのGdドープ量別でのXANESスペクトル}
  \label{fig:Fe-K}
\end{figure}

図\ref{fig:Fe-K2}はGd-6と標準試料$\ce{\alpha-Fe2O3},\ce{Fe3O4},\ce{FeO}$のXANESスペクトルを示している。価数は高エネルギー側にシフトすると
上がるため、2価である$\ce{FeO}$の吸収端エネルギーから高エネルギー側にシフトしていくと、3価の\ce{\alpha-Fe2O3}
吸収端エネルギーが現れる。
2価の$\ce{Fe}$と3価の$\ce{Fe}$を1:2で有する$\ce{Fe3O4}$は、二つの吸収端エネルギーの中間に存在する。作製した$\ce{Gd}$ドープMn-Zn ferriteは、\ce{\alpha-Fe2O3}と
同じ位置に吸収端エネルギーが見られるため、$\ce{Fe}$の価数は3価であることが分かる。\\

吸収端エネルギーの前の$E = \qty{7114}{eV}$付近に見られるピークはpre-edge peakと呼ばれる\cite{NihonXAFS2025}。
通常の吸収端エネルギーと呼ばれるピークは、電気双極子遷移の選択律$\Delta l = \pm1$に従って、K殻の電子がX線を吸収して$1s$軌道から$4p$軌道に遷移するときに生じる。今回のpre-edge peakは電気四重極子遷移によるものと、混成軌道によるものの二つで考えられる。一つ目は電気双極子遷移の選択律に反して電子が$1s$軌道から$3d$軌道へ電気四重極子遷移するためであると考えられる。二つ目は$d$-$p$による混合した軌道の$p$成分への電気双極子遷移するためであると考えられる。これら二つの遷移により、作製したサンプルは結晶の対称性が低いと考えられる。
Gd-6と似た挙動を示している$\alpha-\ce{Fe2O3}$はこの二つによるpre-edge peakと報告されているため、Gd-6でも同じことが起きていると考えられる\cite{Naveas2023}。
\begin{figure}[htbp]
  \centering
  \includegraphics[width=0.8\textwidth]{Fe-K.pdf}
  \caption{Fe K吸収端でのGd-6と標準試料でのXANESスペクトル}
  \label{fig:Fe-K2}
\end{figure}

図\ref{fig:Mn-K}は、作製したサンプルの$\ce{Gd}$ドープ量別に加えて、\ce{Gd}9\%の試料を\ce{Ar}なしで焼成したサンプルはGd-9NoArとし、これらのXANESスペクトルを示している。Mn K吸収端においてGd-9NoArでは吸収端が鋭いピークを示さなかった。これは\ce{Ar}なしの焼成により\ce{Mn}が高価数化、局所構造が乱れたためである。
他の$\ce{Gd}$ドープ量別のスペクトルではシフトは見られなかった。これは$\ce{Gd}$ドープによる$\ce{Mn}$の価数の変化の影響はないことが示された。pre-edge peakはFe K吸収端と同様の現象が起きていると考えられる。。

図\ref{fig:Zn-K}は作製したサンプルの$\ce{Gd}$ドープ量別のXANESスペクトルを示している。Zn K吸収端エネルギーのシフトが見られないため、
$\ce{Gd}$ドープによる$\ce{Zn}$の価数の変化の影響はないことが示された\cite{NihonXAFS2025}。\\
\begin{figure}[htbp]
  \centering
  \includegraphics[width=0.7\textwidth]{Mn-K.pdf}
  \caption{Mn K吸収端でのGdドープ量別でのXANESスペクトル}
  \label{fig:Mn-K}
\end{figure}

\begin{figure}[H]
  \centering
  \includegraphics[width=0.7\textwidth]{Zn-K.pdf}
  \caption{Zn K吸収端でのGdドープ量別でのXANESスペクトル}
  \label{fig:Zn-K}
\end{figure}

\newpage

\subsection{磁化測定}
\subsubsection{Gdドープ量別磁化($M$-$H$)曲線}
\begin{figure}[htbp]
  \centering
  \includegraphics[width=0.9\textwidth]{Gd0-20_MH.png}
  \caption{Gdドープ量が異なる各サンプルの磁化($M$-$H$)曲線}
  \label{fig:M-H}
\end{figure}
\begin{table}[htb]
  \centering
  \caption{$\ce{Gd}$ドープ量が異なる各サンプルの磁化パラメータ}\label{tbl:MH}
   \begin{tabular}{lccc}
    \hline
     サンプル名 & 飽和磁化$M_s$ (\si{emu~ g^{-1}}) & 初透磁率$\mu_i$ (\si{emu~ cm^{-3} ~Oe^{-1}}) & 保磁力$H_c$  (\si{Oe}) \\ 
     \hline
     Gd-0 (\SI{14.6}{nm}) & 48.57 & 6.185 & -5.75 \\
     Gd-6 (\SI{14.1}{nm}) & 48.26 & 6.756 & -5.87 \\
     Gd-9 (\SI{14.1}{nm}) & 42.31 & 6.275 & -5.66 \\
     Gd-20 (\SI{14.4}{nm}) & 40.82 & 5.215 & -7.07 \\ 
     \hline
  \end{tabular}
\end{table}

作製したMn-Zn ferriteナノ微粒子への$\ce{Gd}$ドープ別の磁気特性を調べるため磁化($M$-$H$)測定を行った。図\ref{fig:M-H}は$\ce{Gd}$ドープ量別での磁化曲線を
示している。表\ref{tbl:MH}はMHループから飽和磁化、初透磁率、保磁力を算出した結果である。まずサンプルの密度を計算するために
表\ref{tbl:XRF1}の組成式より計算した分子量を$M$、PDXL2より算出した格子定数$a$から計算した。よって初透磁率は式\eqref{eq:透磁率}で表せる。
\begin{equation}
  \mu_i = 1 + 4 \pi \frac{M Z}{N_Aa^3} \chi_\text{mass}
  \label{eq:透磁率}
\end{equation}
ここで、$Z$は単位格子あたりにある化学式数、$N_A$はアボガドロ数、$\chi_\text{mass}$は質量初磁化率であり、
今回はSQUID磁束計において、印加した磁場の$1$点目\qty{0}{Oe}と$2$点目\qty{25}{Oe}の磁化曲線の傾きを用いている。また、保磁力$H_c$は
\begin{equation}
  H_c = \frac{H_{C+} + H_{C-}}{2}
  \label{保磁力}
\end{equation}
で求めており、$H_{C+}$は磁場を\qty{1}{T}から\qty{-1}{T}に変化させたときの磁化がゼロになる磁場強度、
$H_{C-}$は磁場を\SI{-1}{T}から\SI{1}{T}に変化させたときの、磁化がゼロになる磁場強度である。

表\ref{tbl:MH}より飽和磁化は、$\ce{Gd}$をドープすることによりGd-6まで増加の傾向が見られた。これは$\ce{Gd}$そのものが、高い有効磁気モーメントを持つためであると考えられる。初透磁率は軌道角運動量が0である\ce{Gd}をドープにより、増加したと考えられる。
Gd-9以降では前述したように、不純物の生成により純サンプルの質量が低くなるため、その分減少したと考えられる。飽和磁化は初透磁率と同じ傾向が見られた。
保磁力は、すべてのサンプルにおいて負の値を示している。これはサンプル作製時に粒径の差が生じることで、強磁性相と超常磁性相の相互作用が起き、逆ヒステリシスループが起きたためであると考えられる。次節で詳しく述べる。
\subsubsection{GdドープMn-Zn ferriteの逆ヒステリシス現象}
作製した磁気ナノ微粒子において、$M$-$H$測定を行ったところ図\ref{fig:M-H}において逆ヒステリシス現象を観測した。逆ヒステリシスという現象
は図\ref{fig:逆}のような通常とは逆回転のヒステリシスループを示す現象である。$M$-$H$測定において逆ヒステリシスループが起こっていることを
式\eqref{逆ヒステリシス1}を用いて説明できる。
\begin{figure}[htb]
  \centering
  \includegraphics[width=0.9\textwidth]{逆ヒステリシス.pdf}
  \caption{逆ヒステリシス曲線}
  \label{fig:逆}
\end{figure}
\begin{equation}\label{逆ヒステリシス1}
 \Delta M = M_1 - M_2
\end{equation}
ここで$M_1$は外部磁場が\qty{-1}{T}から\qty{1}{T}印加したときの\qty{0}{Oe}での磁化、
$M_2$は外部磁場が\qty{1}{T}から\qty{-1}{T}印加したときの
\qty{0}{Oe}での磁化である。通常のヒステリシスループでは$\Delta M$は正になるが、逆ヒステリシスループでは負の値をとる。
実際にGd-0、Gd-6、Gd-9での$\Delta M$を計算したところ図\ref{fig:M-H_1kOe}のような結果を示し、測定したサンプルにおいては逆ヒステリシスループを確認した。\\
\begin{figure}[htb]
  \centering
  \includegraphics[width=0.9\textwidth]{MH_sa_1kOe.pdf}
  \caption{Gdドープ量別の各サンプルの$\Delta M$を図\ref{fig:M-H}算出した結果をグラフ化したもの}
  \label{fig:M-H_1kOe}
\end{figure}
逆ヒステリシスループを確認したサンプルは表\ref{tbl:逆ヒステリシス}に残留磁化と保磁力を示した。
\begin{table}[htb]
  \centering
  \caption{Gdドープ量別のサンプルの残留磁化$M_r$と保磁力$H_c$を図\ref{fig:M-H}から算出した結果}\label{tbl:逆ヒステリシス}
   \begin{tabular}{lccc}
    \hline
     サンプル名 & 残留磁化$M_\text{r}$ & 保磁力$H_c$ \\
     \hline
     Gd-0 (\SI{14.6}{nm}) & -0.48 & -5.75 \\
     Gd-6 (\SI{14.1}{nm}) & -0.49 & -5.87 \\
     Gd-9 (\SI{14.1}{nm}) & -0.38 & -5.66 \\ 
     \hline
  \end{tabular}
\end{table}
逆ヒステリシスループ特有の負の残留磁化と負の保磁力を示した。また、このようなループは過去の研究においても、同様の挙動が見られている\cite{Yang2008,Gu2014}。

この逆ヒステリシスループの現象は、超常磁性相と強磁性相の反強磁性的相互作用による現象と考えることができる。これは粒径の差によるものと
粒子内部と表面での磁気異方性が異なることによって起こる二つの機構がある\cite{Yang2008,Gu2014}。

一つ目の機構の粒径の差による2つの磁性相の出現について考える。単磁区における磁気異方性エネルギー$E_\text{B}$はStoner-Wohlfarthのモデルに基づいて式\eqref{eq:K}で表すことができる\cite{A1938}。
\begin{equation}\label{eq:K}
 E_\text{b} = K_\text{eff}V_\text{sample}\sin^2{\theta}
\end{equation}
ここで$\theta$は磁化と容易軸のなす角、$K_\text{eff}$は有効磁気異方性定数、$V_\text{sample}$はサンプルの体積である。
式\eqref{eq:K}よりサンプルの粒径が大きくなると、磁気異方性エネルギー$E_\text{b}$が増加し熱エネルギー$k_\text{B}T$を
超えることができなくなる。これを強磁性相と呼ぶ。粒径が小さくなると、磁気異方性が下がり$E_\text{b}<k_\text{B}T$によって自由な方向に磁化することができる。
これを超常磁性相と呼ぶ。今回計測した$M$-$H$ループでは磁場\qty{1}{T}を印加し、すべての粒子の磁化を飽和させた。
その後磁場を弱めていく過程で、強磁性相は外部磁場の方向に磁化が保たれた状態であるため、強磁性相が作る反磁場により、
周辺にある超常磁性相が負の方向に磁化してしまう。この超常磁性相の総磁化が強磁性相の総磁化を上回ることで負の残留磁化を示す。
これにより磁気異方性の差が粒子間相互作用を起こし逆ヒステリシスループになることを説明できる\cite{Yang2008}。\\

もう一つの機構は、粒子表面では格子欠陥や歪みがあるため磁気異方性が大きくなり、内部ではその歪みなどが少ないため磁気異方性が小さいと
考えられている\cite{Gu2014}。先ほどとの機構の違いは粒子間で起こるか、粒子内部で起こるかの違いであり、負の残留磁化を示す仕組みは同じである。
本研究で作製したサンプルはどちらかの機構で説明できると考えられる。そこで逆ヒステリシスループを確認したサンプルにおいて
\qty{-50}{Oe}から\qty{50}{Oe}と\qty{-100}{Oe}から\qty{100}{Oe}の低磁場領域で$M$-$H$ループを測定することで、強磁性相が磁化の方向に
固定されない状態の$M$-$H$ループを確認した。さらにそのときの$\Delta M$を計算した。二つの結果は図\ref{fig:MH-50}と図\ref{fig:MH-100}に示した。
$\Delta M$が常に正を示しているため、これは正のヒステリシスループを示していることが示された。この結果から、低磁場領域では磁気異方性が大きな粒子は、磁化の方向がランダムになるため磁化が固定されず、相互作用がはたらかない。高磁場領域では、磁気異方性が大きい粒子は強磁性相の磁化が固定されるため、反強磁性的相互作用がはたらき、先ほど説明した逆ヒステリシスループが、作製したサンプルでも起きているという裏付けになると考えられる。

さらに逆ヒステリシスループにおける残留磁化の温度依存性を調べるため、残留磁化についての測定を行った。測定においては最初に磁場を\qty{1}{T}印加し、磁場を弱めていき\qty{0}{Oe}になった時の磁化を
、残留磁化$M_r$として温度別に計測した。結果は図\ref{fig:Mr-T}のようになった。
さらに高温側から低温側にいく過程で残留磁化が負から正になるときがある。この時の温度を反転温度$T_r$とし、各サンプルの結果を表\ref{tbl:Mr-T}のようにした。
\begin{table}[htbp]
  \centering
  \caption{\ce{Gd}ドープ量別反転温度$T_r$}\label{tbl:Mr-T}
   \begin{tabular}{lccc}
    \hline
     サンプル名 & 反転温度(K) \\
     \hline
     Gd-0 (\SI{14.6}{nm}) & 218.97 \\
     Gd-6 (\SI{7.5}{nm}) & 91.16 \\
     Gd-6 (\SI{10.5}{nm}) & 142.91 \\ 
     Gd-6 (\SI{14.1}{nm}) & 218.61 \\
     Gd-6 (\SI{16.9}{nm}) & 261.94 \\ 
     Gd-6 (\SI{23.5}{nm}) & -\\ 
     \hline
  \end{tabular}
\end{table}
反転温度$T_r$以上では前述したように逆ヒステリシスループが起きる。しかし反転温度以下では、熱エネルギーが磁気異方性エネルギーを下回るため、
超常磁性相の粒子が少なくなってしまう。よって強磁性相のつくる反磁界の影響を受ける粒子が減り、粒子の\qty{0}{Oe}での総磁化は正となるためである。

Gd-6(\qty{7.5}{nm})とGd-6(\qty{14.1}{nm})のサンプルにおいて温度別で$M$-$H$ループを測定した。反転温度以下では正のヒステリシスループを示し残留磁化と保磁力は正を示した、反転温度以上では
負のヒステリシスループを示し残留磁化と保磁力は負の値を示した。
\begin{figure}[H]
  \centering
  \includegraphics[width=0.9\textwidth]{50Oe.pdf}
  \caption{(a)\qty{-50}{Oe}から\qty{50}{Oe}でのGd-6のM-Hループとそこから算出した$\Delta M$(b)\qty{-50}{Oe}から\qty{50}{Oe}でのGd-0のM-Hループとそこから算出した$\Delta M$}
  \label{fig:MH-50}
\end{figure}

\begin{figure}[htbp]
  \centering
  \includegraphics[width=0.9\textwidth]{100Oe.pdf}
  \caption{(a)\qty{-100}{Oe}から\qty{100}{Oe}でのGd-6の$M$-$H$ループとそこから算出した$\Delta M$(b)\qty{-100}{Oe}から\qty{100}{Oe}でのGd-0の$M$-$H$ループとそこから算出した$\Delta M$}
  \label{fig:MH-100}
\end{figure}

\begin{figure}[htbp]
  \centering
  \includegraphics[width=0.8\textwidth]{Mr-T.pdf}
  \caption{(a)Gd-0の温度別の残留磁化$M_r$(b)--(f)Gd-6の粒径別の各サンプルの温度別の残留磁化$M_r$}
  \label{fig:Mr-T}
\end{figure}

\begin{figure}[htbp]
  \centering
  \includegraphics[width=0.9\textwidth]{MH.pdf}
  \caption{温度別Gd-0とGd-6の$M$-$H$ループ(a)$T = 5K$での$M$-$H$ループ(b)$T = 100 K$での$M$-$H$ループ(c)$T = 300 K$での$M$-$H$ループ}
  \label{fig:MH}
\end{figure}

\clearpage

\subsubsection{交流磁化の温度依存性(AC-T)測定}
作製した磁気ナノ微粒子の交流磁場下での温度依存性を、調べるためSQUID磁束計を用いて測定を行った。ここで交流磁場は、$h(t) = h_0 \cos(2\pi ft)$
と正弦波で表現することができ、磁気ナノ微粒子は磁化に対して遅れて反応するため、複素磁化率で表すことができる。
\begin{align}
  m(t)
  &= \chi^{\mathrm{(AC)}} h(t) \notag \\
  &= \chi h_0 \cos(2\pi f t + \phi) \notag \\
  &= \chi' h_0 \cos(2\pi f t) - \chi'' h_0 \sin(2\pi f t) \\
  \therefore \quad
  \chi^{\mathrm{(AC)}} &= \chi' - i\chi''
  \label{eq:ACsuscep}
\end{align}
ここで、$\chi'$は交流磁化率の実数部であり同位相の磁化を表す。$\chi''$は交流磁化率の虚数部であり遅れた磁化を表す。
Gd-6の粒径別($10.5,14.1,16.9,\qty{23.5}{nm}$)の測定結果は図\ref{fig:AC}である。Gd-6の\qty{14.1}{nm}のサンプルが最も高い$\chi'$を示した。
これは最大の初透磁率を持つため、交流磁化に対する応答が良好であると考えられ、MPIシグナルにおいても体温(\qty{310}{K})付近においては
\qty{14.1}{nm}のサンプルが最も大きなMPIシグナルを示すと考えられる。
\begin{figure}[htbp]
  \centering
  \includegraphics[width=1.0\textwidth]{AC.pdf}
  \caption{(a)Gd-6での温度別交流磁化率の実数部$\chi'$(b)Gd-0での温度別交流磁化率の虚数部$\chi''$}
  \label{fig:AC}
\end{figure}

\newpage

\subsection{MPIシグナル測定}
作製した磁気ナノ微粒子のMPI応用に向けて、MPIシグナルである第三高調波応答を計測した。ここでMPIシグナルの強度として、下式のように計算を行った。
\begin{equation}
  V_\textrm{3h} = V_\textrm{3h, raw} - V_\textrm{BG}
\end{equation}
ここで、$V_\textrm{3h}$は第三高調波、$V_\textrm{3h, raw}$はサンプルを用いた測定時の第三高調波強度、$V_\textrm{BG}$は
空サンプルを用いたバックグラウンド測定時の第三高調波強度である。

作製した\ce{Gd}ドープMn-Zn ferriteの\ce{Gd}ドープ量別の、MPIシグナルを測定した。測定では駆動磁場周波数\qty{500}{Hz}、駆動磁場強度\qty{150}{Oe}
の交流磁化率を印加し、それに対してのサンプルの磁化応答を、MPIシグナルとして測定した。ここでのサンプルはXRDによりスピネル構造を確認した
サンプルを用いた。粒径は\qtyrange{13}{15}{nm}の範囲で粒径を揃えている。測定結果を示した図から、Gd-6のサンプルが最も大きなシグナルを示すことが示された。
これはドープ量6\%が最も高い初透磁率と高い飽和磁化を持つため、磁化応答が良いと考えられる。Gd-6以降のMPIシグナルの減少は、不純物の生成により磁化率が小さくなるためだと考えられる。先行研究の\ce{Gd}ドープMn ferriteにおいても、不純物を生成しない高い初透磁率を持つサンプルが、最も高いMPIシグナルを示している\cite{Sakamoto2024}。よって不純物が生成せず、初透磁率の高い\ce{Gd}ドープ量がMPIシグナルにおいて最適であると考えられる。
\begin{figure}[htbp]
  \centering
  \includegraphics[width=0.7\textwidth]{MPI_Gd.png}
  \caption{Gdドープ量別のMPIシグナル}
  \label{fig:MPI}
\end{figure}

\newpage

\subsection{磁気共鳴イメージング(MRI)}
MPI応用ではトレーサーが存在する部分の撮像に特化しているため、MRIとの撮像を併用することでさまざまな病変を見つけることができる。
MRI画像は体内の水素原子中のプロトン\ce{^1H}を利用することで見ることができる。さらに造影剤を用いることで、水素原子中のプロトンの緩和時間の減少を促すことができ、
それによりコントラストの強い画像を得ることができる\cite{Frey2009}。

作製した磁気ナノ微粒子について$T_2$緩和測定を行った。測定方法はスピンエコー法を用いて、Repetition Time(TR)\qty{2500}{ms}に設定した。
$T_2$緩和測定の結果から\ce{Gd}を入れたことによる緩和時間$T_2$の短縮は見られたが、図のような強いノイズが発生したことにより、定量的に計算することが困難である。

作製した磁気ナノ微粒子について$T_1$緩和測定を行った。Echo Time(TE)\qty{2500}{ms}に固定し測定を行った。
$T_1$緩和曲線から緩和率$R_1$を計算した結果を図に示した。寒天と比べて大幅に緩和率$R_1$を増加させた。これは先行研究であるGdドープMn ferriteと同じ挙動を示した。これは磁気モーメントが大きい\ce{Gd}により
磁気ナノ微粒子と水素分子中のプロトンの相互作用が、増強されたことによると考えられる。粒径による緩和率$R_1$の差は、飽和磁化の大きさによるものと、
磁気ナノ微粒子とプロトンとの平均距離によるもの、二つの要素が影響していると考えられる。これらの結果から、超常磁性の高い飽和磁化を持った
\ce{Gd}ドープMn-Zn ferriteを使うことは、MRI造影剤での$T_1$強調画像において、より鮮明な画像を得られることが期待される。
\begin{figure}[htbp]
  \centering
  \includegraphics[width=1.0\textwidth]{T1T2_Gd.pdf}
  \caption{(a)寒天とGd-6の粒径が異なる各サンプルの$T_2$緩和曲線(b)寒天とGd-6の粒径が異なる各サンプルの$T_1$緩和曲線}
  \label{fig:MRIT}
\end{figure}
\begin{figure}[htbp]
  \centering
  \includegraphics[width=0.7\textwidth]{0211MRI_R1.png}
  \caption{寒天とGd-6の粒径が異なる各サンプルの緩和率$R_1$}
  \label{fig:MRIT1}
\end{figure}

\subsection{アミノ基修飾磁気ナノ微粒子}
作製した磁気ナノ微粒子に細胞選択性を持たせることで、特定の病変に対してMPI、MRIを撮像することができる。そこでGd-6の表面に、
アミノ基を修飾できるかを確認するため実験を行った。
Gd-6にTEOS、APTES、アンモニア水溶液を用いて24時間攪拌させた後、遠心分離により洗浄を行った。この時、遊離\ce{-NH2}基が存在しないことを
ニンヒドリン試験により確認した。乾燥させたGd-6にニンヒドリン試験を行ったところ、図のように紫色になったことを確認した。
これにより作製したサンプルはアミノ基修飾が可能であることを確認した。

\begin{figure}[htbp]
  \centering
  \includegraphics[width=0.4\textwidth]{ニンヒドリン反応.jpg}
  \caption{(左)純水にニンヒドリン溶液をいれたもの(右)アミノ基修飾をしたGd-6にニンドリ溶液を入れたもの}
  \label{fig:ニンヒドリン反応}
\end{figure}





\section{結論}
本研究ではイメージング用の新規造影剤として、高い磁化と低い磁気異方性を持つ\ce{Gd}をドープしたMn-Zn ferriteについて研究を行ってきた。研究室独自の湿式混合法により\ce{Gd}ドープMn-Zn ferriteの作製をした。

作製した全てのサンプルがXRFにより組成式$\ce{Mn_{0.8-x}Zn_{0.2}Gd_{x}Fe2O4}$
通りになっていることを確認した。さらにXRDの解析により、作製したサンプルがスピネル構造に見られるピークと同じものを確認した。\ce{Gd}ドープ量\qty{9}{\%}においては不純物相のピークが見られ、\ce{Gd}ドープによる不純物の現れない、最適な\ce{Gd}ドープ量を特定できた。さらに二つのピークの解析から、\ce{Gd}ドープ量\qty{8}{\%}まで格子定数が増加していることを確認した。これはイオン半径の大きい\ce{Gd}が構造内にドープされていることを裏付ける解析結果を得た。XAFS解析からは\ce{Gd}ドープによる価数の変化は見られないことを確認した。

磁化測定からは\ce{Gd}ドープ量を\qty{6}{\%}に調整した試料が最も高い初透磁率と高い飽和磁化を持つことを確認した。さらに作製したサンプルにおいて、逆ヒステリシス現象を確認した。この現象は低温での$M$-$H$ループ、残留磁化の温度依存性について実験を行うことでさらに解析を行なった。これらの測定から作製した粒子には、大きな粒径、小さな粒径が混在し、磁気異方性の大小で強磁性相と超常磁性相の2つの磁性相が生まれる。これにより$T = \qty{300}{\kelvin}$、高磁場領域では逆ヒステリシスループが生まれたと考える。

\ce{Gd}ドープ量別のMPIシグナルにおいては、\ce{Gd}ドープ量\qty{6}{\%}のサンプルが最も高い応答強度を示した。これは最大の初透磁率と高い飽和磁化を持つことによるものだと考える。MRI測定においては\ce{Gd}ドープにより、飽和磁化の高いサンプルがプロトンの磁気緩和時間を下げたことを確認した。表面修飾実験においてはアミノ基修飾をつけることに成功した。これらの結果により、磁気粒子イメージングにおいて\ce{Gd}ドープMn-Zn ferriteは、MPI、MRI測定においての新規造影剤として期待される。さらに表面修飾により選択性を持つ磁気ナノ微粒子としての可能性が示された。

\bibliographystyle{naturemag} 
\bibliography{MK6}

\section{謝辞}
本研究を進めるにあたり、様々なご指導をいただきました一柳優子教授に深く感謝申し上げます。 

また、研究室で共に研究を進めてきた楠本悠羽さん、渡邉将太郎さん、川井楓さん、砂川遼太さん、星川直輝さんには、多くの議論交わし、様々な助言や励ましをいただきました。心より感謝申し上げます。特に、実験や測定のサポート、様々な助言を多くしていただきました楠本悠羽さん、共に議論を交わし研究を進めた星川直輝さん、実験や測定のサポートをして頂きました渡邉将太郎さん、重ねて感謝申し上げます。

また本研究において、RINT2500を用いたXRD測定につきましては、横浜国立大学 教育人間学部 津野宏先生にご協力いただきました。XAFS測定につきましては、高エネルギー加速器研究機構 物質構造科学研究所 関係者の皆様にご協力いただきました。MRI測定につきましては、東京大学 大学院工学系研究科 バイオエンジニアリング先行 関野正樹先生にご協力いただきました。SQUID磁束計を用いた磁化測定につきましては、東京科学大学 総合研究院 川路均先生、木谷卓先生、大阪大学 熱 エントロピー科学研究センター 中野元裕先生、宮崎裕司先生、中沢康浩先生、横浜市立大学 国際総合学部 山田重樹先生にご協力いただきました。各大学、研究機関の関係者の皆様に深く感謝申し上げます。

厚く御礼申し上げるとともに、これを謝辞と代えさせていただきます。

\subsection{本研究に関わる研究費助成一覧}
また、本研究は以下の研究費助成を受けて実施されました。

\begin{itemize}
  \renewcommand{\labelitemi}{・}
  \item KEK 放射光共同利用実験課題 2024G600
  「Gd, Zn 共ドープ Mn-Zn ferrite 系ナノ微粒子における金属原子の配位特性と局所構造解析」
  一柳優子,2024--2026 年採択
  \item 日本学術振興会 科学研究費助成事業 基盤研究 (B)
  「超常磁性スピンクラスターの磁気緩和現象の解明と創薬への応用」
  一柳優子,2025--2027 年採択
  \item 高橋経済研究財団 研究助成 255
  「がん細胞選択性を持つ磁気ナノ微粒子の開発」
  一柳優子,2025 年度採択
  \item YNU 国際ネットワークハブ
  「ナノ物性物理とバイオの融合研究拠点」
  一柳優子,2024--2026 年採択
\end{itemize}

\section{業績}
\begin{itemize}
  \renewcommand{\labelitemi}{・}
  \item 18th International Sinposium on Nanomedicine (ISNM2025) (2025.12.1-3広島大学広仁会館)
ポスター番号 P-01
「Optimization of Magnetic Properties and MPI signals of Gd-Doped Mn-Zn Ferrite Nanoparticles」
K. Miura, Y. Kusumoto, N. Hoshikawa, S. Watanabe, and Y. Ichiyanagi
\end{itemize}



\end{document}
\section{結果及び考察}
\subsection{\ce{Gd}ドープMn-Zn ferriteの作製}
\ce{Gd}ドープMn-Zn ferriteは組成式$\ce{Mn_{0.8-x}Zn_{0.2}Gd_xFe2O4}$を目標に湿式混合法により作製した。
本研究で作製したサンプルを表\ref{tbl:Sample}に示す。
\begin{table}
  \caption{\textbf{本研究で作製したGdドープ量別サンプルの一覧}}\label{tbl:Sample}
  \centering
  \begin{tabular}{l|ccccccccccc}
    \hline
    サンプル名 & Gd-0 & Gd-1 & Gd-2 & Gd-3 & Gd-4 & Gd-5 & Gd-6 & 
    Gd-7 & Gd-8 & Gd-9 & Gd-20 \\
    \hline
    Gd含有量$x$ & 0 & 0.01 & 0.02 & 0.03 & 0.04 & 0.05 & 0.06 & 0.07 & 0.08 & 0.09 & 0.2 \\
    \hline
    粒径(nm) & 14.2 & 14.3 & 14.5 & 14.3 & 14.4 & 14.3 & 14.5 & 14.4 & 14.3 & 14.2 & 13.0 \\
    \hline
  \end{tabular}
\end{table}

\subsection{蛍光X線分析(XRF)}
本研究で作製した$\ce{Gd}$ドープMn-Zn ferriteナノ微粒子のXRF分析を行い得られたピークからFP法を行い各元素のモル比率を定量化した。
表\ref{tbl:XRF}は得られた各サンプルのモル比率である。\\
\begin{table}
  \caption{\textbf{蛍光X線分析で測定したGdドープMn-Zn ferriteナノ微粒子のモル\%}}\label{tbl:XRF}
  \centering
  \begin{tabular}{l|ccccc}
    \hline
    サンプル名 & Mn & Zn & Gd & Fe & Si\\
    \hline
    Gd-0 & 19.47(3) & 4.47(7) & 0 & 47.96(3) & 28.10(20)\\
    Gd-1 & 18.88(4) & 4.00(8) & 0.28(12) & 45.60(38) & 31.24(21) \\
    Gd-2 & 18.55(4) & 4.03(7) & 0.46(11) & 45.50(4) & 31.46(20)\\
    Gd-3 & 19.10(6) & 5.03(11) & 0.62(16) & 49.64(5) & 25.60(29)\\
    Gd-4 & 17.89(4) & 4.04(8) & 0.95(12) & 45.43(4) & 31.68(21)\\
    Gd-5 & 18.20(4) & 4.00(7) & 1.05(10) & 48.14(3) & 28.61(18)\\
    Gd-6 & 18.08(4) & 4.37(7) & 1.27(10) & 47.58(3) & 28.70(18)\\
    Gd-7 & 17.71(4) & 4.11(7) & 1.59(11) & 47.73(3) & 28.86(19)\\
    Gd-8 & 17.38(4) & 4.56(7) & 1.81(11) & 48.48(3) & 27.76(19)\\
    Gd-9 & 18.04(4) & 4.25(7) & 2.04(11) & 50.34(3) & 25.33(20)\\
    Gd-20 & 14.16(4) & 3.60(7) & 4.53(11) & 41.88(3) & 35.83(19)\\
    \hline
  \end{tabular}
\end{table}

\newpage

また表のモル比率から作製したサンプルの組成比を調べるために下式を使って計算した。\\
\begin{equation}
  X_M=\frac{3x_M}{x_{\ce{Mn}}+x_{\ce{Zn}}+x_{\ce{Gd}}+x_{\ce{Fe}}}
\end{equation}
ここで$X_M$は求めたい組成比、$x_M$は求めたい原子のモル比率、$x_{\ce{Mn}},x_{\ce{Zn}},x_{\ce{Gd}},x_{\ce{Fe}}$は各金属原子のモル比率である。
計算の結果は表\ref{tbl:XRF1}のようになった。

作製したサンプルは概ね秤量値通りの目的の組成比で作製することができた。

\begin{table}
  \caption{\textbf{蛍光X線分析で測定したGdドープMn-Zn ferriteナノ微粒子の組成}}\label{tbl:XRF1}
  \centering
  \begin{tabular}{l|ccccc}
    \hline
    サンプル名 & Mn & Zn & Gd & Fe & Si\\
    \hline
    Gd-0 & 0.812 & 0.187 & 0 & 2.001 & 1.172\\
    Gd-1 & 0.824 & 0.175 & 0.012 & 1.990 & 1.363 \\
    Gd-2 & 0.812 & 0.176 & 0.020 & 1.992 & 1.377\\
    Gd-3 & 0.770 & 0.203 & 0.025 & 2.002 & 1.032\\
    Gd-4 & 0.786 & 0.177 & 0.042 & 1.995 & 1.391\\
    Gd-5 & 0.765 & 0.168 & 0.044 & 2.023 & 1.202\\
    Gd-6 & 0.761 & 0.184 & 0.053 & 2.002 & 1.208\\
    Gd-7 & 0.747 & 0.173 & 0.067 & 2.013 & 1.217\\
    Gd-8 & 0.722 & 0.189 & 0.075 & 2.014 & 1.153\\
    Gd-9 & 0.725 & 0.171 & 0.082 & 2.022 & 1.018\\
    Gd-20 & 0.646 & 0.176 & 0.206 & 1.972 & 1.319\\
    \hline
  \end{tabular}
\end{table}

\subsection{粉末X線回折(XRD)}
\subsubsection{Gdドープ量別Mn-Zn ferriteのXRD}

$\ce{Gd}$ドープ量の異なる各サンプルのXRDパターンを、図\ref{fig:XRD}に示した。粒径は\qtyrange{13}{15}{nm}に揃えた。
全てのサンプルにおいて、XRD測定は連続法で測定を行った。
全てのサンプルにおいて、スピネル構造に対応するピークが確認されており、Fd-3mの空間群で指数付けすることができた。
また、$\theta = $\qtyrange{15}{30}{\degree}付近において、アモルファス$\ce{SiO2}$由来のブロードなピークが確認され、作製時にアモルファス$\ce{SiO2}$
が合成されていることがわかった。粒径の算出は、最も強いピーク強度を持つ(311)面において、FP法を適用することで、求めた値を採用している。\\
\begin{figure}[H]
  \centering
  \includegraphics[width=0.9\textwidth]{XRD_Gd0-20.pdf}
  \caption{Gdドープ量が異なる各サンプルのXRDパターン}
  \label{fig:XRD}
\end{figure}

Gd-20に関しては、図\ref{fig:XRDGd-20}からスピネル構造には見られないピークが観察されており、これは$\ce{Mn2Gd8(SiO4)6O2}$に対応するものと考えられる。六方晶構造に対応するピークが確認され、空間群はP$6_3/$mで指数付けをすることができた。
一般に希土類イオン(\ce{RE^{3+}})のスピネルフェライトへのドープでは、\ce{Fe^{3+}}と比較してイオン半径が大きいためスピネル格子への固溶限界が低いことが知られている。この固溶限界を超えると、斜方晶のferriteなどの二次相が生成することが報告されている\cite{Sharma2021}。
本研究におけるGd-20の試料作製では、組成比$\ce{Mn_{0.8-x}Zn_{0.2}Gd_xFe2O4}$を揃えるため、\ce{Gd}の秤量値を補正のため13\%ほど増やし調整している。さらに\ce{Gd}はイオン半径が大きくスピネル構造に入りにくいと考えられ、過剰に存在した\ce{Gd}がスピネル相として固溶せず、焼成過程において共存する$\ce{SiO2}$と反応した結果、$\ce{Mn2Gd8(SiO4)6O2}$が二次相として生成したものと考えられる。\\
\begin{figure}[H]
  \includegraphics[width=0.9\textwidth]{XRD_Gd-20.png}
  \caption{Gd-20のXRDパターン}
  \label{fig:XRDGd-20}
\end{figure}

また\ce{Gd}ドープによるピーク位置のシフトについて調べた。Braggの式から格子面間隔$d$が大きくなると、回折角度$\theta$は小さくなる\cite{Bragg1913}。
\begin{equation}
  2d \sin \theta = n \lambda
\end{equation}
ここで$d$は格子面間隔、$\theta$は回折角度、$n$は回折次数、$\lambda$はX線の波長である。今回作製したサンプルの各金属元素のイオン半径は以下のようになる。
表\ref{tbl:イオン半径}に示すように、$\ce{Gd}^{3+}$のイオン半径は\ce{Mn}、\ce{Zn}、\ce{Fe}の金属イオンより大きいため、
\ce{Gd}ドープ量が増加するにつれて低角側にシフトすると考えられる\cite{Shannon1976}。

\begin{table}[H]
  \centering
  \caption{\ce{Mn},\ce{Zn},\ce{Gd},\ce{Fe}のイオン半径(HS:高スピン、LS:低スピン)}\label{tbl:イオン半径}
   \begin{tabular}{lcccc}
    \hline
    元素 & 元素名 & 価数 (スピン状態) & 配位数 & 有効イオン半径(\AA) \\ \hline
     Mn & マンガン & 2+ & 4 & 0.66 \\
     Mn & マンガン & 2+ & 6 & 0.830 \\
     Zn & 亜鉛 & 2+  & 4 & 0.60 \\
     Gd & ガドリニウム & 3+  & 6 & 0.938 \\
     Fe & 鉄 & 3+ (HS) & 4 & 0.49 \\
     Fe & 鉄 & 3+ (LS) & 6 & 0.55 \\
     Fe & 鉄 &  3+ (HS) & 6 & 0.645 \\ \hline
  \end{tabular}
\end{table}
そこで最も回折強度の大きな(311)面におけるピーク位置と、
広角側の比較的大きな(440)面のピークを角度分解能を上げて測定をした。測定条件は角度分解能を上げ、FixedTime(FT)法で測定した。
\begin{figure}[htbp]
  \centering
  \includegraphics[width=1.0\textwidth]{XRDfitting.pdf}
  \caption{角度分解能を上げた$\ce{Gd}$ドープ量別でのXRDパターン(a)測定範囲$2\theta =$ \qtyrange{33}{37}{\degree}の(311)面のピーク(b)$2\theta =$ \qtyrange{60}{64}{\degree}の(440)面のピーク}
  \label{fig:XRD_peak}
\end{figure}
図\ref{fig:XRD_peak}が測定結果である。測定したデータはノイズが大きく、ピークの位置を判断することが困難である。そこで
測定データを、擬フォークト関数(ガウス分布関数とローレンツ分布関数の畳み込み)を使いピーク角度を検出した\cite{Balzar1993}。
擬フォークト関数は以下の式\eqref{eq:voigt}で表される。フィッティングした結果が図\ref{fig:XRDfitting}である。
\begin{equation}\label{eq:voigt}
y = y_0 + A \left[
m_u \frac{2}{\pi} \frac{w}{4\left(x - x_c\right)^2 + w^2}
+ \left(1 - m_u\right)
\frac{\sqrt{4\ln 2}}{\sqrt{\pi}\, w}
\exp\!\left(
-\frac{4\ln 2}{w^2}\left(x - x_c\right)^2
\right)
\right]
\end{equation}
ここで$y_0$はオフセット、$x_c$は中心位置、$A$は面積、$w$はガウス分布関数とフォークト関数のピーク強度の半分の高さにおける幅(FWHM)、$m_\text{u}$はプロファイル形状係数である。
\begin{figure}[H]
  \centering
  \includegraphics[width=1.0\textwidth]{fitting.pdf}
  \caption{図\ref{fig:XRD_peak}を擬フォークト関数によりフィッティングしたXRDパターン(a)(311)面のピーク(b)(440)面のピーク}
  \label{fig:XRDfitting}
\end{figure}
さらに$\ce{Gd}$ドープ量別の(311)面と(440)面のピーク角度から格子定数を算出した結果が図\ref{fig:lattice}である。
\begin{figure}[htbp]
  \centering
  \includegraphics[width=1.0\textwidth]{lattice.pdf}
  \caption{図\ref{fig:XRDfitting}から格子定数を算出しグラフ化したもの(a)(311)面のピーク(b)(440)面のピーク}
  \label{fig:lattice}
\end{figure}
これより$\ce{Gd}$-0から$\ce{Gd}$-8までピークが低角側にシフトしていることが確認でき、
Mn-Zn ferriteにイオン半径の大きい$\ce{Gd}$がドープされ、格子定数が増加したと説明できる。当研究室の先行研究である\ce{Gd}ドープMn ferriteと同じ挙動を示した\cite{Sakamoto2024}。
$\ce{Gd}$-9と$\ce{Gd}$-20に関してはピーク位置のシフトが広角側にシフトしている。これは$\ce{Gd}$過剰ドープにより、前述したように$\ce{Mn2Gd8(SiO4)6O2}$の六方晶構造ができ、$\ce{Gd}$が構造に入らず、
目的の試料は$\ce{Mn_{0.8-x}Zn_{0.2}Gd_{x}Fe2O4}$であるため、\ce{Gd}のドープ量が増加することで、$\ce{Mn}$の割合が減少し、相対的に$\ce{Zn}$の割合が増加する。$\ce{Zn}$イオンは他の金属イオンよりイオン半径が小さいため
格子定数が小さくなり、ピーク位置が広角側にシフトしたと考えられる。またPDXL2を用いて、ピーク角度の算出を行い格子定数を算出した結果は、図\ref{fig:lattice_PDXL}のようになり同じ傾向を示した。\\
\begin{figure}[H]
  \centering
  \includegraphics[width=0.7\textwidth]{lattice_PDXL.png}
  \caption{図\ref{fig:XRD_peak}をPDXL2より算出した格子定数}
  \label{fig:lattice_PDXL}
\end{figure}
\subsubsection{粒径別GdドープMn-Zn ferriteのXRD}
作製したGd-6のサンプルを、焼成温度を調整することで粒径\qtyrange{7.5}{23.5}{nm}の範囲で、6つのサンプルを作製した。これらの粒径別の各サンプルを\ce{Gd}ドープ量別で行ったXRDの測定条件と同様に測定を行った。全ての粒径別のサンプルにおいて図\ref{fig:Gd-6_size}に示した結果から、\ce{SiO2}のブロードなピークが見られ、スピネル構造を確認した。不純物相のピークは見られなかった。\\
\begin{figure}[H]
  \centering
  \includegraphics[width=0.6\textwidth]{Gd-6_粒径.png}
  \caption{Gd-6粒径別の各サンプルのXRDパターン}
  \label{fig:Gd-6_size}
\end{figure}

\newpage

\subsection{X線吸収微細構造測定(XAFS)}
\begin{figure}[H]
  \centering
  \includegraphics[width=0.8\textwidth]{XAFS.png}
  \caption{XAFSスペクトル}
  \label{fig:XAFS}
\end{figure}
作製した磁気ナノ微粒子の価数を調べるため、X線吸収微細構造(XAFS)測定を用いて構造解析を行った。作製した$\ce{Gd}$ドープ量別のサンプルを
\ce{Mn}、\ce{Zn}、\ce{Fe}のK吸収端において測定した。XAFSスペクトルは、吸収端付近のX-ray absorption near edge structure(XANES)スペクトルが得られた。
XANESスペクトルからは、電子状態などの中心原子の状態に依存する情報を得ることができる。そこで吸収端エネルギーのシフトを見ることで、各金属元素の価数を調べ、さらにpre-edge peakから結晶の対称性を調べた。
XAFSスペクトルの解析は、XAFS解析ソフトウェア(Athena)を用いて解析を行った。測定データのスペクトルをpre-edge lineとpost-edge lineを、それぞれ1次関数と3次関数を使って差し引いた。そこからpre-edge lineが0、EXAFS領域が1を中心に振動するように規格化を行った。

ここで、EXAFS領域については、今回作製したサンプルが金属元素の種類が多く、解析が複雑になってしまう。EXAFS領域の解析にあたりフィッティングを行うが、その結果の信頼性を次の式から定量的に調べる方法が知られている\cite{NihonXAFS2025}。
\begin{equation}\label{eq:Nyquist}
N_\text{ind} = \frac{2\Delta k \Delta R}{\pi} + m
\end{equation}
ここで$N_\text{ind}$は独立点の数、$\Delta k$はフーリエ変換の範囲、$\Delta R$は逆フーリエ変換の範囲、mは0,1,2のいずれかである。作製したサンプルはスピネル構造のため、最近接原子の全ては酸素原子であるため、第2配位圏の金属原子を調べる必要がある。ここで$\Delta k = 12$、$\Delta R = 2$とすると$N_\text{ind}$は16とし、中心元素
としてA-siteに入っている\ce{Zn}のパラメータ数を計算してみる。散乱経路数はA-siteの金属原子に\ce{Mn}、\ce{Zn}、\ce{Gd}、\ce{Fe}の4種類、B-siteの金属原子に\ce{Mn}、\ce{Gd}、\ce{Fe}の3種類で散乱経路数は7である。さらに各散乱経路に対して、最低でも3つのパラメータが必要であるため、パラメータ数は21となる。これより$N_\text{ind}$よりパラメータ数が多くなってしまうため、EXAFS領域の解析は行わない。

図\ref{fig:Fe-K}は、作製したサンプルの$\ce{Gd}$ドープ量別のXANESスペクトルを示し、Fe K吸収端エネルギーのシフトが見られないため、
$\ce{Gd}$ドープによる$\ce{Fe}$の価数の影響はないとわかる。\\
\begin{figure}[H]
  \centering
  \includegraphics[width=0.8\textwidth]{Fe-K_Gd.pdf}
  \caption{Fe K吸収端でのGdドープ量別でのXANESスペクトル}
  \label{fig:Fe-K}
\end{figure}

図\ref{fig:Fe-K2}はGd-6と標準試料$\ce{\alpha-Fe2O3},\ce{Fe3O4},\ce{FeO}$のXANESスペクトルを示している。価数は高エネルギー側にシフトすると
上がるため、2価である$\ce{FeO}$の吸収端エネルギーから高エネルギー側にシフトしていくと、3価の\ce{\alpha-Fe2O3}
吸収端エネルギーが現れる。
2価の$\ce{Fe}$と3価の$\ce{Fe}$を1:2で有する$\ce{Fe3O4}$は、二つの吸収端エネルギーの中間に存在する。作製した$\ce{Gd}$ドープMn-Zn ferriteは、\ce{\alpha-Fe2O3}と
同じ位置に吸収端エネルギーが見られるため、$\ce{Fe}$の価数は3価であることが分かる。\\

吸収端エネルギーの前の$E = \qty{7114}{eV}$付近に見られるピークはpre-edge peakと呼ばれる\cite{NihonXAFS2025}。
通常の吸収端エネルギーと呼ばれるピークは、電気双極子遷移の選択律$\Delta l = \pm1$に従って、K殻の電子がX線を吸収して$1s$軌道から$4p$軌道に遷移するときに生じる。今回のpre-edge peakは電気四重極子遷移によるものと、混成軌道によるものの二つで考えられる。一つ目は電気双極子遷移の選択律に反して電子が$1s$軌道から$3d$軌道へ電気四重極子遷移するためであると考えられる。二つ目は$d$-$p$による混合した軌道の$p$成分への電気双極子遷移するためであると考えられる。これら二つの遷移により、作製したサンプルは結晶の対称性が低いと考えられる。
Gd-6と似た挙動を示している$\alpha-\ce{Fe2O3}$はこの二つによるpre-edge peakと報告されているため、Gd-6でも同じことが起きていると考えられる\cite{Naveas2023}。
\begin{figure}[htbp]
  \centering
  \includegraphics[width=0.8\textwidth]{Fe-K.pdf}
  \caption{Fe K吸収端でのGd-6と標準試料でのXANESスペクトル}
  \label{fig:Fe-K2}
\end{figure}

図\ref{fig:Mn-K}は、作製したサンプルの$\ce{Gd}$ドープ量別に加えて、\ce{Gd}9\%の試料を\ce{Ar}なしで焼成したサンプルはGd-9NoArとし、これらのXANESスペクトルを示している。Mn K吸収端においてGd-9NoArでは吸収端が鋭いピークを示さなかった。これは\ce{Ar}なしの焼成により\ce{Mn}が高価数化、局所構造が乱れたためである。
他の$\ce{Gd}$ドープ量別のスペクトルではシフトは見られなかった。これは$\ce{Gd}$ドープによる$\ce{Mn}$の価数の変化の影響はないことが示された。pre-edge peakはFe K吸収端と同様の現象が起きていると考えられる。。

図\ref{fig:Zn-K}は作製したサンプルの$\ce{Gd}$ドープ量別のXANESスペクトルを示している。Zn K吸収端エネルギーのシフトが見られないため、
$\ce{Gd}$ドープによる$\ce{Zn}$の価数の変化の影響はないことが示された\cite{NihonXAFS2025}。\\
\begin{figure}[htbp]
  \centering
  \includegraphics[width=0.7\textwidth]{Mn-K.pdf}
  \caption{Mn K吸収端でのGdドープ量別でのXANESスペクトル}
  \label{fig:Mn-K}
\end{figure}

\begin{figure}[H]
  \centering
  \includegraphics[width=0.7\textwidth]{Zn-K.pdf}
  \caption{Zn K吸収端でのGdドープ量別でのXANESスペクトル}
  \label{fig:Zn-K}
\end{figure}

\newpage

\subsection{磁化測定}
\subsubsection{Gdドープ量別磁化($M$-$H$)曲線}
\begin{figure}[htbp]
  \centering
  \includegraphics[width=0.9\textwidth]{Gd0-20_MH.png}
  \caption{Gdドープ量が異なる各サンプルの磁化($M$-$H$)曲線}
  \label{fig:M-H}
\end{figure}
\begin{table}[htb]
  \centering
  \caption{$\ce{Gd}$ドープ量が異なる各サンプルの磁化パラメータ}\label{tbl:MH}
   \begin{tabular}{lccc}
    \hline
     サンプル名 & 飽和磁化$M_s$ (\si{emu~ g^{-1}}) & 初透磁率$\mu_i$ (\si{emu~ cm^{-3} ~Oe^{-1}}) & 保磁力$H_c$  (\si{Oe}) \\ 
     \hline
     Gd-0 (\SI{14.6}{nm}) & 48.57 & 6.185 & -5.75 \\
     Gd-6 (\SI{14.1}{nm}) & 48.26 & 6.756 & -5.87 \\
     Gd-9 (\SI{14.1}{nm}) & 42.31 & 6.275 & -5.66 \\
     Gd-20 (\SI{14.4}{nm}) & 40.82 & 5.215 & -7.07 \\ 
     \hline
  \end{tabular}
\end{table}

作製したMn-Zn ferriteナノ微粒子への$\ce{Gd}$ドープ別の磁気特性を調べるため磁化($M$-$H$)測定を行った。図\ref{fig:M-H}は$\ce{Gd}$ドープ量別での磁化曲線を
示している。表\ref{tbl:MH}はMHループから飽和磁化、初透磁率、保磁力を算出した結果である。まずサンプルの密度を計算するために
表\ref{tbl:XRF1}の組成式より計算した分子量を$M$、PDXL2より算出した格子定数$a$から計算した。よって初透磁率は式\eqref{eq:透磁率}で表せる。
\begin{equation}
  \mu_i = 1 + 4 \pi \frac{M Z}{N_Aa^3} \chi_\text{mass}
  \label{eq:透磁率}
\end{equation}
ここで、$Z$は単位格子あたりにある化学式数、$N_A$はアボガドロ数、$\chi_\text{mass}$は質量初磁化率であり、
今回はSQUID磁束計において、印加した磁場の$1$点目\qty{0}{Oe}と$2$点目\qty{25}{Oe}の磁化曲線の傾きを用いている。また、保磁力$H_c$は
\begin{equation}
  H_c = \frac{H_{C+} + H_{C-}}{2}
  \label{保磁力}
\end{equation}
で求めており、$H_{C+}$は磁場を\qty{1}{T}から\qty{-1}{T}に変化させたときの磁化がゼロになる磁場強度、
$H_{C-}$は磁場を\SI{-1}{T}から\SI{1}{T}に変化させたときの、磁化がゼロになる磁場強度である。

表\ref{tbl:MH}より飽和磁化は、$\ce{Gd}$をドープすることによりGd-6まで増加の傾向が見られた。これは$\ce{Gd}$そのものが、高い有効磁気モーメントを持つためであると考えられる。初透磁率は軌道角運動量が0である\ce{Gd}をドープにより、増加したと考えられる。
Gd-9以降では前述したように、不純物の生成により純サンプルの質量が低くなるため、その分減少したと考えられる。飽和磁化は初透磁率と同じ傾向が見られた。
保磁力は、すべてのサンプルにおいて負の値を示している。これはサンプル作製時に粒径の差が生じることで、強磁性相と超常磁性相の相互作用が起き、逆ヒステリシスループが起きたためであると考えられる。次節で詳しく述べる。
\subsubsection{GdドープMn-Zn ferriteの逆ヒステリシス現象}
作製した磁気ナノ微粒子において、$M$-$H$測定を行ったところ図\ref{fig:M-H}において逆ヒステリシス現象を観測した。逆ヒステリシスという現象
は図\ref{fig:逆}のような通常とは逆回転のヒステリシスループを示す現象である。$M$-$H$測定において逆ヒステリシスループが起こっていることを
式\eqref{逆ヒステリシス1}を用いて説明できる。
\begin{figure}[htb]
  \centering
  \includegraphics[width=0.9\textwidth]{逆ヒステリシス.pdf}
  \caption{逆ヒステリシス曲線}
  \label{fig:逆}
\end{figure}
\begin{equation}\label{逆ヒステリシス1}
 \Delta M = M_1 - M_2
\end{equation}
ここで$M_1$は外部磁場が\qty{-1}{T}から\qty{1}{T}印加したときの\qty{0}{Oe}での磁化、
$M_2$は外部磁場が\qty{1}{T}から\qty{-1}{T}印加したときの
\qty{0}{Oe}での磁化である。通常のヒステリシスループでは$\Delta M$は正になるが、逆ヒステリシスループでは負の値をとる。
実際にGd-0、Gd-6、Gd-9での$\Delta M$を計算したところ図\ref{fig:M-H_1kOe}のような結果を示し、測定したサンプルにおいては逆ヒステリシスループを確認した。\\
\begin{figure}[htb]
  \centering
  \includegraphics[width=0.9\textwidth]{MH_sa_1kOe.pdf}
  \caption{Gdドープ量別の各サンプルの$\Delta M$を図\ref{fig:M-H}算出した結果をグラフ化したもの}
  \label{fig:M-H_1kOe}
\end{figure}
逆ヒステリシスループを確認したサンプルは表\ref{tbl:逆ヒステリシス}に残留磁化と保磁力を示した。
\begin{table}[htb]
  \centering
  \caption{Gdドープ量別のサンプルの残留磁化$M_r$と保磁力$H_c$を図\ref{fig:M-H}から算出した結果}\label{tbl:逆ヒステリシス}
   \begin{tabular}{lccc}
    \hline
     サンプル名 & 残留磁化$M_\text{r}$ & 保磁力$H_c$ \\
     \hline
     Gd-0 (\SI{14.6}{nm}) & -0.48 & -5.75 \\
     Gd-6 (\SI{14.1}{nm}) & -0.49 & -5.87 \\
     Gd-9 (\SI{14.1}{nm}) & -0.38 & -5.66 \\ 
     \hline
  \end{tabular}
\end{table}
逆ヒステリシスループ特有の負の残留磁化と負の保磁力を示した。また、このようなループは過去の研究においても、同様の挙動が見られている\cite{Yang2008,Gu2014}。

この逆ヒステリシスループの現象は、超常磁性相と強磁性相の反強磁性的相互作用による現象と考えることができる。これは粒径の差によるものと
粒子内部と表面での磁気異方性が異なることによって起こる二つの機構がある\cite{Yang2008,Gu2014}。

一つ目の機構の粒径の差による2つの磁性相の出現について考える。単磁区における磁気異方性エネルギー$E_\text{B}$はStoner-Wohlfarthのモデルに基づいて式\eqref{eq:K}で表すことができる\cite{A1938}。
\begin{equation}\label{eq:K}
 E_\text{b} = K_\text{eff}V_\text{sample}\sin^2{\theta}
\end{equation}
ここで$\theta$は磁化と容易軸のなす角、$K_\text{eff}$は有効磁気異方性定数、$V_\text{sample}$はサンプルの体積である。
式\eqref{eq:K}よりサンプルの粒径が大きくなると、磁気異方性エネルギー$E_\text{b}$が増加し熱エネルギー$k_\text{B}T$を
超えることができなくなる。これを強磁性相と呼ぶ。粒径が小さくなると、磁気異方性が下がり$E_\text{b}<k_\text{B}T$によって自由な方向に磁化することができる。
これを超常磁性相と呼ぶ。今回計測した$M$-$H$ループでは磁場\qty{1}{T}を印加し、すべての粒子の磁化を飽和させた。
その後磁場を弱めていく過程で、強磁性相は外部磁場の方向に磁化が保たれた状態であるため、強磁性相が作る反磁場により、
周辺にある超常磁性相が負の方向に磁化してしまう。この超常磁性相の総磁化が強磁性相の総磁化を上回ることで負の残留磁化を示す。
これにより磁気異方性の差が粒子間相互作用を起こし逆ヒステリシスループになることを説明できる\cite{Yang2008}。\\

もう一つの機構は、粒子表面では格子欠陥や歪みがあるため磁気異方性が大きくなり、内部ではその歪みなどが少ないため磁気異方性が小さいと
考えられている\cite{Gu2014}。先ほどとの機構の違いは粒子間で起こるか、粒子内部で起こるかの違いであり、負の残留磁化を示す仕組みは同じである。
本研究で作製したサンプルはどちらかの機構で説明できると考えられる。そこで逆ヒステリシスループを確認したサンプルにおいて
\qty{-50}{Oe}から\qty{50}{Oe}と\qty{-100}{Oe}から\qty{100}{Oe}の低磁場領域で$M$-$H$ループを測定することで、強磁性相が磁化の方向に
固定されない状態の$M$-$H$ループを確認した。さらにそのときの$\Delta M$を計算した。二つの結果は図\ref{fig:MH-50}と図\ref{fig:MH-100}に示した。
$\Delta M$が常に正を示しているため、これは正のヒステリシスループを示していることが示された。この結果から、低磁場領域では磁気異方性が大きな粒子は、磁化の方向がランダムになるため磁化が固定されず、相互作用がはたらかない。高磁場領域では、磁気異方性が大きい粒子は強磁性相の磁化が固定されるため、反強磁性的相互作用がはたらき、先ほど説明した逆ヒステリシスループが、作製したサンプルでも起きているという裏付けになると考えられる。

さらに逆ヒステリシスループにおける残留磁化の温度依存性を調べるため、残留磁化についての測定を行った。測定においては最初に磁場を\qty{1}{T}印加し、磁場を弱めていき\qty{0}{Oe}になった時の磁化を
、残留磁化$M_r$として温度別に計測した。結果は図\ref{fig:Mr-T}のようになった。
さらに高温側から低温側にいく過程で残留磁化が負から正になるときがある。この時の温度を反転温度$T_r$とし、各サンプルの結果を表\ref{tbl:Mr-T}のようにした。
\begin{table}[htbp]
  \centering
  \caption{\ce{Gd}ドープ量別反転温度$T_r$}\label{tbl:Mr-T}
   \begin{tabular}{lccc}
    \hline
     サンプル名 & 反転温度(K) \\
     \hline
     Gd-0 (\SI{14.6}{nm}) & 218.97 \\
     Gd-6 (\SI{7.5}{nm}) & 91.16 \\
     Gd-6 (\SI{10.5}{nm}) & 142.91 \\ 
     Gd-6 (\SI{14.1}{nm}) & 218.61 \\
     Gd-6 (\SI{16.9}{nm}) & 261.94 \\ 
     Gd-6 (\SI{23.5}{nm}) & -\\ 
     \hline
  \end{tabular}
\end{table}
反転温度$T_r$以上では前述したように逆ヒステリシスループが起きる。しかし反転温度以下では、熱エネルギーが磁気異方性エネルギーを下回るため、
超常磁性相の粒子が少なくなってしまう。よって強磁性相のつくる反磁界の影響を受ける粒子が減り、粒子の\qty{0}{Oe}での総磁化は正となるためである。

Gd-6(\qty{7.5}{nm})とGd-6(\qty{14.1}{nm})のサンプルにおいて温度別で$M$-$H$ループを測定した。反転温度以下では正のヒステリシスループを示し残留磁化と保磁力は正を示した、反転温度以上では
負のヒステリシスループを示し残留磁化と保磁力は負の値を示した。
\begin{figure}[H]
  \centering
  \includegraphics[width=0.9\textwidth]{50Oe.pdf}
  \caption{(a)\qty{-50}{Oe}から\qty{50}{Oe}でのGd-6のM-Hループとそこから算出した$\Delta M$(b)\qty{-50}{Oe}から\qty{50}{Oe}でのGd-0のM-Hループとそこから算出した$\Delta M$}
  \label{fig:MH-50}
\end{figure}

\begin{figure}[htbp]
  \centering
  \includegraphics[width=0.9\textwidth]{100Oe.pdf}
  \caption{(a)\qty{-100}{Oe}から\qty{100}{Oe}でのGd-6の$M$-$H$ループとそこから算出した$\Delta M$(b)\qty{-100}{Oe}から\qty{100}{Oe}でのGd-0の$M$-$H$ループとそこから算出した$\Delta M$}
  \label{fig:MH-100}
\end{figure}

\begin{figure}[htbp]
  \centering
  \includegraphics[width=0.8\textwidth]{Mr-T.pdf}
  \caption{(a)Gd-0の温度別の残留磁化$M_r$(b)--(f)Gd-6の粒径別の各サンプルの温度別の残留磁化$M_r$}
  \label{fig:Mr-T}
\end{figure}

\begin{figure}[htbp]
  \centering
  \includegraphics[width=0.9\textwidth]{MH.pdf}
  \caption{温度別Gd-0とGd-6の$M$-$H$ループ(a)$T = 5K$での$M$-$H$ループ(b)$T = 100 K$での$M$-$H$ループ(c)$T = 300 K$での$M$-$H$ループ}
  \label{fig:MH}
\end{figure}

\clearpage

\subsubsection{交流磁化の温度依存性(AC-T)測定}
作製した磁気ナノ微粒子の交流磁場下での温度依存性を、調べるためSQUID磁束計を用いて測定を行った。ここで交流磁場は、$h(t) = h_0 \cos(2\pi ft)$
と正弦波で表現することができ、磁気ナノ微粒子は磁化に対して遅れて反応するため、複素磁化率で表すことができる。
\begin{align}
  m(t)
  &= \chi^{\mathrm{(AC)}} h(t) \notag \\
  &= \chi h_0 \cos(2\pi f t + \phi) \notag \\
  &= \chi' h_0 \cos(2\pi f t) - \chi'' h_0 \sin(2\pi f t) \\
  \therefore \quad
  \chi^{\mathrm{(AC)}} &= \chi' - i\chi''
  \label{eq:ACsuscep}
\end{align}
ここで、$\chi'$は交流磁化率の実数部であり同位相の磁化を表す。$\chi''$は交流磁化率の虚数部であり遅れた磁化を表す。
Gd-6の粒径別($10.5,14.1,16.9,\qty{23.5}{nm}$)の測定結果は図\ref{fig:AC}である。Gd-6の\qty{14.1}{nm}のサンプルが最も高い$\chi'$を示した。
これは最大の初透磁率を持つため、交流磁化に対する応答が良好であると考えられ、MPIシグナルにおいても体温(\qty{310}{K})付近においては
\qty{14.1}{nm}のサンプルが最も大きなMPIシグナルを示すと考えられる。
\begin{figure}[htbp]
  \centering
  \includegraphics[width=1.0\textwidth]{AC.pdf}
  \caption{(a)Gd-6での温度別交流磁化率の実数部$\chi'$(b)Gd-0での温度別交流磁化率の虚数部$\chi''$}
  \label{fig:AC}
\end{figure}

\newpage

\subsection{MPIシグナル測定}
作製した磁気ナノ微粒子のMPI応用に向けて、MPIシグナルである第三高調波応答を計測した。ここでMPIシグナルの強度として、下式のように計算を行った。
\begin{equation}
  V_\textrm{3h} = V_\textrm{3h, raw} - V_\textrm{BG}
\end{equation}
ここで、$V_\textrm{3h}$は第三高調波、$V_\textrm{3h, raw}$はサンプルを用いた測定時の第三高調波強度、$V_\textrm{BG}$は
空サンプルを用いたバックグラウンド測定時の第三高調波強度である。

作製した\ce{Gd}ドープMn-Zn ferriteの\ce{Gd}ドープ量別の、MPIシグナルを測定した。測定では駆動磁場周波数\qty{500}{Hz}、駆動磁場強度\qty{150}{Oe}
の交流磁化率を印加し、それに対してのサンプルの磁化応答を、MPIシグナルとして測定した。ここでのサンプルはXRDによりスピネル構造を確認した
サンプルを用いた。粒径は\qtyrange{13}{15}{nm}の範囲で粒径を揃えている。測定結果を示した図から、Gd-6のサンプルが最も大きなシグナルを示すことが示された。
これはドープ量6\%が最も高い初透磁率と高い飽和磁化を持つため、磁化応答が良いと考えられる。Gd-6以降のMPIシグナルの減少は、不純物の生成により磁化率が小さくなるためだと考えられる。先行研究の\ce{Gd}ドープMn ferriteにおいても、不純物を生成しない高い初透磁率を持つサンプルが、最も高いMPIシグナルを示している\cite{Sakamoto2024}。よって不純物が生成せず、初透磁率の高い\ce{Gd}ドープ量がMPIシグナルにおいて最適であると考えられる。
\begin{figure}[htbp]
  \centering
  \includegraphics[width=0.7\textwidth]{MPI_Gd.png}
  \caption{Gdドープ量別のMPIシグナル}
  \label{fig:MPI}
\end{figure}

\newpage

\subsection{磁気共鳴イメージング(MRI)}
MPI応用ではトレーサーが存在する部分の撮像に特化しているため、MRIとの撮像を併用することでさまざまな病変を見つけることができる。
MRI画像は体内の水素原子中のプロトン\ce{^1H}を利用することで見ることができる。さらに造影剤を用いることで、水素原子中のプロトンの緩和時間の減少を促すことができ、
それによりコントラストの強い画像を得ることができる\cite{Frey2009}。

作製した磁気ナノ微粒子について$T_2$緩和測定を行った。測定方法はスピンエコー法を用いて、Repetition Time(TR)\qty{2500}{ms}に設定した。
$T_2$緩和測定の結果から\ce{Gd}を入れたことによる緩和時間$T_2$の短縮は見られたが、図のような強いノイズが発生したことにより、定量的に計算することが困難である。

作製した磁気ナノ微粒子について$T_1$緩和測定を行った。Echo Time(TE)\qty{2500}{ms}に固定し測定を行った。
$T_1$緩和曲線から緩和率$R_1$を計算した結果を図に示した。寒天と比べて大幅に緩和率$R_1$を増加させた。これは先行研究であるGdドープMn ferriteと同じ挙動を示した。これは磁気モーメントが大きい\ce{Gd}により
磁気ナノ微粒子と水素分子中のプロトンの相互作用が、増強されたことによると考えられる。粒径による緩和率$R_1$の差は、飽和磁化の大きさによるものと、
磁気ナノ微粒子とプロトンとの平均距離によるもの、二つの要素が影響していると考えられる。これらの結果から、超常磁性の高い飽和磁化を持った
\ce{Gd}ドープMn-Zn ferriteを使うことは、MRI造影剤での$T_1$強調画像において、より鮮明な画像を得られることが期待される。
\begin{figure}[htbp]
  \centering
  \includegraphics[width=1.0\textwidth]{T1T2_Gd.pdf}
  \caption{(a)寒天とGd-6の粒径が異なる各サンプルの$T_2$緩和曲線(b)寒天とGd-6の粒径が異なる各サンプルの$T_1$緩和曲線}
  \label{fig:MRIT}
\end{figure}
\begin{figure}[htbp]
  \centering
  \includegraphics[width=0.7\textwidth]{0211MRI_R1.png}
  \caption{寒天とGd-6の粒径が異なる各サンプルの緩和率$R_1$}
  \label{fig:MRIT1}
\end{figure}

\subsection{アミノ基修飾磁気ナノ微粒子}
作製した磁気ナノ微粒子に細胞選択性を持たせることで、特定の病変に対してMPI、MRIを撮像することができる。そこでGd-6の表面に、
アミノ基を修飾できるかを確認するため実験を行った。
Gd-6にTEOS、APTES、アンモニア水溶液を用いて24時間攪拌させた後、遠心分離により洗浄を行った。この時、遊離\ce{-NH2}基が存在しないことを
ニンヒドリン試験により確認した。乾燥させたGd-6にニンヒドリン試験を行ったところ、図のように紫色になったことを確認した。
これにより作製したサンプルはアミノ基修飾が可能であることを確認した。

\begin{figure}[htbp]
  \centering
  \includegraphics[width=0.4\textwidth]{ニンヒドリン反応.jpg}
  \caption{(左)純水にニンヒドリン溶液をいれたもの(右)アミノ基修飾をしたGd-6にニンドリ溶液を入れたもの}
  \label{fig:ニンヒドリン反応}
\end{figure}




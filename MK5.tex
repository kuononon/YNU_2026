\section{結論}
本研究ではイメージング用の新規造影剤として、高い磁化と低い磁気異方性を持つ\ce{Gd}をドープしたMn-Zn ferriteについて研究を行ってきた。研究室独自の湿式混合法により\ce{Gd}ドープMn-Zn ferriteの作製をした。

作製した全てのサンプルがXRFにより組成式$\ce{Mn_{0.8-x}Zn_{0.2}Gd_{x}Fe2O4}$
通りになっていることを確認した。さらにXRDの解析により、作製したサンプルがスピネル構造に見られるピークと同じものを確認した。\ce{Gd}ドープ量\qty{9}{\%}においては不純物相のピークが見られ、\ce{Gd}ドープによる不純物の現れない、最適な\ce{Gd}ドープ量を特定できた。さらに二つのピークの解析から、\ce{Gd}ドープ量\qty{8}{\%}まで格子定数が増加していることを確認した。これはイオン半径の大きい\ce{Gd}が構造内にドープされていることを裏付ける解析結果を得た。XAFS解析からは\ce{Gd}ドープによる価数の変化は見られないことを確認した。

磁化測定からは\ce{Gd}ドープ量を\qty{6}{\%}に調整した試料が最も高い初透磁率と高い飽和磁化を持つことを確認した。さらに作製したサンプルにおいて、逆ヒステリシス現象を確認した。この現象は低温での$M$-$H$ループ、残留磁化の温度依存性について実験を行うことでさらに解析を行なった。これらの測定から作製した粒子には、大きな粒径、小さな粒径が混在し、磁気異方性の大小で強磁性相と超常磁性相の2つの磁性相が生まれる。これにより$T = \qty{300}{\kelvin}$、高磁場領域では逆ヒステリシスループが生まれたと考える。

\ce{Gd}ドープ量別のMPIシグナルにおいては、\ce{Gd}ドープ量\qty{6}{\%}のサンプルが最も高い応答強度を示した。これは最大の初透磁率と高い飽和磁化を持つことによるものだと考える。MRI測定においては\ce{Gd}ドープにより、飽和磁化の高いサンプルがプロトンの磁気緩和時間を下げたことを確認した。表面修飾実験においてはアミノ基修飾をつけることに成功した。これらの結果により、磁気粒子イメージングにおいて\ce{Gd}ドープMn-Zn ferriteは、MPI、MRI測定においての新規造影剤として期待される。さらに表面修飾により選択性を持つ磁気ナノ微粒子としての可能性が示された。
\tableofcontents

\newpage

\section{諸元}
\subsection{研究背景}
\subsubsection{ナノテクノロジー}
ナノテクノロジーとはナノスケール(\qtyrange{1}{100}{nm})での加工、操作、制御を指し、医療分野やエレクトロニクス
における応用が期待される技術である\cite{Kim2008}。当研究室では、ナノサイズ化した磁性ナノ微粒子を応用した研究が進められてきた\cite{Sakamoto2023,Aoki2022}。
磁気ナノ微粒子は特異的な磁気特性を持ち、磁気イメージングにおいて、
本研究でも扱う磁気粒子イメージング(MPI)や、磁気共鳴イメージング(MRI)などが盛んに研究されている\cite{Na2007,Gleich2005}。MPIは、GleichとWeizeneckerにより提唱された
新しいイメージング技術であり、この技術はトレーサーそのもののシグナルを読み取るため、高感度かつ高空間分解能で、磁性ナノ微粒子の分布を可視化できる\cite{Gleich2005}。
MPIに関する研究では、\Resovist を用いた研究が主流である。MRIはBlochとPurcellにより提唱された\cite{Bloch1946,Purcell1946}。
NMR現象を用いた測定であり、水素原子中のプロトン(\ce{^1H})からのシグナルにより、$T_1$強調画像や$T_2$強調画像を得られる。MRI測定では体内の
水素原子中のプロトン(\ce{^1H})を利用するが、特定の病変を画像のコントラストを上げて見るため、Gd-DTPAなどの造影剤を用いて測定されることもある\cite{Na2007}。さらに両手法において
新規造影剤の開発も盛んに行われている。

\subsection{研究目的}
イメージング用の新規造影剤として、高い磁化と低い磁気異方性を持つ、磁性ナノ微粒子が最適であると考えられている。先行研究ではMn ferriteナノ微粒子に、
\ce{Gd}をドープすることで初透磁率が増加することが報告されている\cite{Sakamoto2023}。Mn-Zn ferriteナノ微粒子に関しては、$\ce{Mn_{0.8}Zn_{0.2}Fe2O4}$の組成において、
最も大きな飽和磁化を示し、超常磁性であることが知られている\cite{Kondo2015,Kondo2014}。そこで本研究では、Mn-Zn ferriteに\ce{Gd}をドープし、粒径をナノサイズにすることで
造影剤としての性能評価を目的としている。作製した磁気ナノ微粒子は構造解析、磁気特性解析を行った。さらにアミノ基を導入する実験を行うことで、
選択性を持つ磁気ナノ微粒子としての可能性を評価した。MPI測定では、作製した磁気ナノ微粒子は、\ce{Gd}ドープ量を変えMPIシグナルの最適化を行った。MRI測定では、\ce{Gd}ドープにより水素原子中のプロトンの磁気緩和現象に
どのような影響を与えるかを評価した。以上の研究で得られた結果から、イメージング用新規造影剤としての可能性を深く検討した。


\section{実験}
\subsection{GdドープMn-Zn ferriteナノ微粒子の作製}
GdドープMn-Zn ferriteは下記の化学反応式より湿式混合法(特許第3933366号)を用いて作製した。
\begin{equation}\label{eq:湿式混合法}
  \begin{aligned}
 \ce{(0.8-$x$)MnCl2 + 0.2ZnCl2 + $x$GdCl3 + 2FeCl2 + 3Na2SiO3 + $x$NaOH}\\
 \ce{-> Mn_{0.8-$x$}Zn_{0.2}Gd_{$x$}Fe2O4 + 3SiO2 + {6+$x$}NaCl}
  \end{aligned}
\end{equation}

金属塩とアルカリを純水に溶かしたあと、混合し中和反応を起こし水酸化物沈殿を作製した。
その後、遠心分離を行い洗浄し、乾燥させた試料を粉砕し焼成した。
表\ref{tbl:reagents}に$\ce{Mn_{0.8-x}Zn_{0.2}Gd_{x}Fe2O4}$ナノ微粒子を合成するため以下の試薬をそのまま使用した。\\

\begin{table}[htbp]
\centering
\caption{GdドープMn--Zn ferriteナノ微粒子の作製に用いた試薬}
\label{tbl:reagents}
\begin{tabular}{lll}
\hline
試薬名 & 組成式 & 純度・製造元 \\
\hline
塩化マンガン四水和物 &
$\ce{MnCl2 . 4H2O}$ &
99.9\%, 富士フィルム和光純薬株式会社 \\

塩化亜鉛 &
$\ce{ZnCl2}$ &
98.0\%, 富士フィルム和光純薬株式会社 \\

塩化ガドリニウム六水和物 &
$\ce{GdCl3 . 6H2O}$ &
99.0\%, 富士フィルム和光純薬株式会社 \\

塩化鉄(II)六水和物 &
$\ce{FeCl2 . 6H2O}$ &
99.0\%, 富士フィルム和光純薬株式会社 \\

メタケイ酸ナトリウム九水和物 &
$\ce{Na2SiO3 . 9H2O}$ &
99.9\%, 富士フィルム和光純薬株式会社 \\

水酸化ナトリウム &
$\ce{NaOH}$ &
99.0\%, 富士フィルム和光純薬株式会社 \\
\hline
\end{tabular}
\end{table}

\ce{Gd}ドープMn-Zn ferriteナノ微粒子は、式\eqref{eq:湿式混合法}に示す組成式を目標に、$1/200$ mol--$1/10$ molのサンプルを作製した。
まず純水\qty{50}{mL}に塩化マンガン四水和物、塩化亜鉛、塩化ガドリニウム六水和物、塩化鉄(II)六水和物を溶解させ、金属イオン溶液を調整した。
次に純水\qty{100}{mL}にメタケイ酸ナトリウム九水和物、水酸化ナトリウムを溶解させ、アルカリ溶液を調整した。金属イオン溶液は\ce{Gd}のドープ量$x$に合わせて
秤量した。アルカリ溶液はメタケイ酸ナトリウム九水和物は目的のサンプル\qty{1}{mol}に対して\qty{3}{mol}のアモルファス$\ce{SiO2}$が生成されるように秤量した。
水酸化ナトリウムは塩基不足を補うため、塩化ガドリ二ウム$x$ molに対して$x$ mol秤量した。金属塩とアルカリを混合し、マグネティックスターラーと撹拌子で
\qty{350}{rpm}、室温で15分撹拌した。得られた沈殿物を(PP)製\qty{50}{mL}遠沈管にうつして遠心分離を\qty{3600}{rpm}で3分を2回、15分を1回行った。洗浄後、沈殿物を\qty{50}{\degreeCelsius}の乾燥炉で
2日間以上乾燥させた。この試料を乳鉢で15分間粉砕し、粉末状のものをアズプリ(焼成前駆体)とした。
このアズプリを焼成炉を用いて図\ref{fig:焼成プログラム}に示す焼成プログラムで焼成した。
\begin{figure}[htb]
  \centering
   \includegraphics[width=0.5\textwidth]{焼成プログラム.png}
  \caption{焼成プログラム}\label{fig:焼成プログラム}
\end{figure}
このとき\ce{Mn}、\ce{Gd}が酸化して不純物の生成を防ぐため、\qty{40}{mL/min}の\ce{Ar}を常に流しながら行った。焼成温度$Z$を変えることで粒径を調節できるため、
様々な粒径を\qtyrange{1050}{1200}{\kelvin}の間で様々に変化させることで行った。焼成前のサンプル、焼成後のサンプルは図\ref{fig:sample}に示す。
\begin{figure}[htb]
  \centering
   \includegraphics[width=0.8\textwidth]{サンプルGd.pdf}
  \caption{(a)$\ce{Mn_{0.8-x}Zn_{0.2}Gd_{x}Fe2O4}$のアズプリの写真(b)$\ce{Mn_{0.8-x}Zn_{0.2}Gd_{x}Fe2O4}$の焼成後の写真}\label{fig:sample}
\end{figure}

\subsubsection{粉末X線回折測定(XRD)}
作製した磁気ナノ微粒子の結晶構造を調べるため、粉末X線回折測定はリガク社製のMiniFlex IIとRINT2500を用いて行った。各装置は図\ref{fig:XRD}である。
\begin{figure}[htbp]
  \centering
   \includegraphics[width=0.9\textwidth]{XRD装置.pdf}
  \caption{(a)RINT2500の写真(b)MiniFlex IIの写真}\label{fig:XRD}
\end{figure}
MiniFlex IIでは、Cu-K$\alpha$ 線(波長\qty{1.5406}{Å})を用い、管電圧\qty{30}{kV}、管電流\qty{15}{mA}、測定範囲\qtyrange{10}{80}{\degree}、ステップ幅\qty{0.15}{\degree}、測定速度\qty{2}{\degree/min}の条件で連続測定を行った。
RINT2500では、Cu-K$\alpha$ 線(波長\qty{1.5406}{\angstrom})を用い、管電圧\qty{40}{kV}、管電流\qty{45}{mA}で測定を行った。1つ目の測定は$2\theta = $\qtyrange{10}{80}{\degree}、ステップ幅\qty{0.15}{\degree}、測定速度$\qty{2}{\degree/min}$
の条件で連続測定を行った。二つ目の測定では、スピネル構造のミラー指数(311)(440)のピークの高精度測定を目的として、$2\theta = $\qtyrange{33}{37}{\degree}、\qtyrange{60}{64}{\degree}、ステップ幅\qty{0.004}{\degree}、測定方法は
一定時間内のカウント数を計測するFixed Time(FT)法で行った\cite{Warren1950}。
試料ホルダーは、株式会社リガク製のガラス試料ホルダー(\qtyproduct{20x20x0.5}{mm})を用いた。

XRDパターンの解析は、株式会社リガク製のPDXL2を用いて行った。XRDパターンのデータベースは国際結晶データセンター(ICDD)のPDF-4+を用いた。
また、粒径に関してはFP法によって算出された結晶子サイズとして評価を行った\cite{Cheary1992}。

\subsubsection{蛍光X線分析(XRF)}
作製した磁気ナノ微粒子の各金属の組成比を調べるために、蛍光X線分析(XRF)測定を行った。XRF測定は、横浜国立大学の機器分析センターにある日本電子株式会社製の、
JSX-3100RⅡを用いて行った。測定は専用のカップに\qty{10}{mg}程度入れ測定を行った。\ce{Si}元素を含めた測定では、専用の多孔質フィルムをし密閉することで、真空引きを行った。
測定条件はRhのX線管球を用い、管電圧\qty{30}{kV}、管電流\qty{1}{mA}、最適化係数値\qty{25000}{cps}、測定条件\qty{100}{s}の条件で測定を行った。

\subsubsection{X線吸収微細構造解析(XAFS)}
作製した磁気ナノ微粒子の局所構造解析を行うため、高エネルギー加速器研究機構(KEK)において、フォトンファクトリーのBL-9Cのビームラインを用いてXAFS測定を行った。
作製したサンプルを適した形状に成型するため窒化ホウ素($\ce{BN}$)を混合し、油圧プレス機を用いてペレット状にし、適切なX線吸収強度を示すよう加工した。
XAFS測定には、\ce{Mn}、\ce{Zn}、\ce{Fe}のK吸収端と\ce{Gd}の$L_2$吸収端で測定を行った。XAFS測定によって得たデータの解析は、XAFS解析ソフトウェアAthena
を用いて行った\cite{Ravel2005}。

\subsection{磁化測定}
作製した磁気ナノ微粒子の磁気特性を調べるために超伝導量子干渉装置(SQUID)を用いて行った。
測定装置は大阪大学 大学院理学研究科 附属 熱 · エントロピー科学研究センターにある Quantum Design 社製 MPMS-1S、
東京科学大学 フロンティア材料研究所 川路研究室にある Quantum Design 社製 MPMS-7(AC 測定オプション付き)、
横浜市立大学 理学部 理学科 生命ナノシステム科学研究科 物質システム科学専攻 山田研究室にある Quantum Design 社製 MPMS-XL を用いた。
サンプルの測定を行う際に\qty{100}{Oe}の磁場を印加してセンタリングを行い、全ての測定の前に消磁処理(デガウス)を行った。$T = \qty{300}{\kelvin}$では\qty{100}{Oe}、
それ以下の温度ではサンプルの保磁力が大きくなるため、\qty{1000}{Oe}の磁場強度からデガウスを行った。

SQUID磁束系での磁化測定のため図のようなストローと呼ばれるものを作製した。ゼラチンカプセルに測定サンプルを入れ、その後脱脂綿を詰めることで
サンプルを固定した。そのゼラチンカプセルをストローの中にいれ、さらにカプトンテープを用いてカプセルが動かないように固定した。
最後に全体にピンセットを用いて圧力平衡用の穴を30か所ほど開けた。ゼラチンカプセルに封入したサンプルの質量はXRFでの計測の結果を用いて
、式\eqref{eq:純サンプル質量}により$\ce{SiO2}$を抜いた純サンプル量を計算した。これにより$\ce{Mn_{0.8-x}Zn_{0.2}Gd_{x}Fe2O4}$の質量を計算することができる。
計算の結果は表\ref{tbl:SampleList}に示す。
\begin{figure}[htb]
  \centering
   \includegraphics[width=0.7\textwidth]{ストロー.jpg}
  \caption{SQUID測定のため作成したストローの写真}\label{fig:ストロー}
\end{figure}

\begin{equation}
m_\textrm{pure} = m_\textrm{sample} \frac{X}{X + (A_{\ce{Si}}+2A_{\ce{O}})(3 x_{\ce{Si}}/(x_{\ce{Mn}} + x_{\ce{Zn}} + x_{\ce{Gd}} + x_{\ce{Fe}}))}
\label{eq:純サンプル質量}
\end{equation}
\begin{equation}
X = A_{\ce{Mn}}x_{\ce{Mn}} + A_{\ce{Zn}}x_{\ce{Zn}} + A_{\ce{Gd}}x_{\ce{Gd}} + A_{\ce{Fe}}x_{\ce{Fe}} + 4A_{\ce{O}}
\label{eq純サンプル質量補足}
\end{equation}
ここで$m_\textrm{pure}$は純サンプル質量、$m_\textrm{sample}$は実際のサンプル質量、$A_E$は元素$E$の原子量、$x_E$はXRFで測定した元素$E$のモルパーセントである。
\begin{table}
  \caption{\textbf{本研究で作製したGdドープ量別サンプルの一覧}}\label{tbl:SampleList}
  \centering
  \begin{tabular}{ccc}
    \hline
    サンプル名 & サンプル質量(mg) & 純サンプル質量(mg)\\
    \hline
    Gd-0(\qty{14.6}{nm}) & 21.80 & 16.73\\
    Gd-1(\qty{14.2}{nm}) & 21.38 & 15.82\\
    Gd-2(\qty{14.5}{nm}) & 22.00 & 16.30\\
    Gd-3(\qty{14.8}{nm}) & 22.40 & 17.70\\
    Gd-4(\qty{14.9}{nm}) & 21.60 & 16.00\\
    Gd-5(\qty{14.7}{nm}) & 20.15 & 15.40\\
    Gd-6(\qty{10.5}{nm}) & 19.30 & 14.80\\
    Gd-6(\qty{14.1}{nm}) & 21.81 & 16.70\\
    Gd-6(\qty{16.9}{nm}) & 19.60 & 15.00\\
    Gd-6(\qty{23.5}{nm}) & 21.20 & 16.30\\
    Gd-7(\qty{14.1}{nm}) & 22.10 & 16.90\\
    Gd-8(\qty{14.4}{nm}) & 21.00 & 16.30\\
    Gd-9 (\qty{15.3}{nm}) & 50.00 & 37.10\\
    Gd-20(\qty{13.3}{nm}) & 19.50 & 14.86\\
    \hline
  \end{tabular}
\end{table}

磁化曲線の測定はそれぞれの磁場強度での磁化を測定した。磁場は\qtyrange{0}{100}{Oe}においては初透磁率を見るため\qty{25}{Oe}間隔で印加し、
そこから300,1000,3000,\qty{5000}{Oe}間隔で\qty{1}{T}まで印加した。この間隔を基準にして$\pm \qty{1}{T}$の範囲で測定した。
温度に関しては\qty{5}{\kelvin},\qty{200}{\kelvin},\qty{300}{\kelvin}で測定を行った。

交流磁化の温度依存性(AC-T)測定では温度を変化させ、各温度で周波数\qty{10}{Hz},\qty{100}{Hz},\qty{500}{Hz},\qty{1000}{Hz}で測定した。
温度は\qtyrange{150}{300}{\kelvin}の間で\qty{10}{\kelvin}の間隔で変化させ、\qty{1}{Oe}の磁場強度で印加した。

残留磁化の温度依存性($M_{\text{r}}$-$T$)測定は$M-H$曲線における残留磁化$M_{\text{r}}$のみを測定した。
\qty{1}{T}の磁場をかけて磁化が飽和したあと、\qty{0}{Oe}にした際の磁化を測定した。温度は\qtyrange{300}{80}{\kelvin}まで\qty{20}{\kelvin}間隔、
\qtyrange{80}{5}{\kelvin}まで\qty{18.7}{\kelvin}間隔で測定を行った。

\subsection{磁気ナノ微粒子イメージング(MPI)}
\begin{figure}[H]
  \centering
   \includegraphics[width=1.0\textwidth]{MPI装置1.pdf}
  \caption{MPIシグナルの測定に用いたコイル。(a)交流磁場発生コイル。(b)自作した差動巻きのピックアップコイル。}\label{fig:MPI2}
\end{figure}
\begin{figure}[H]
  \centering
   \includegraphics[width=0.4\textwidth]{MPI装置1.png}
  \caption{冷却ピックアップコイルを用いたMPIシグナルの測定様子}\label{fig:MPI1}
\end{figure}
MPIシグナルとされる第三高調波測定は図に示すような自作の装置を用いて測定を行った。測定装置は、交流磁場発生用コイルとピックアップコイルの2つを、
さらにコイルの熱雑音によるノイズを低減するため\qty{77}{\kelvin}の液体窒素で冷却しながら測定を行った。交流磁場は、図に示す交流磁場発生用コイルに、
エヌエフ回路設計ブロック株式会社製BP4610のバイポーラ電源を用いて発生させた交流電源を流すことで印加した。ここで、サンプルに印加される磁場強度は
バイポーラ電源から直流電流を印加し、その電流における磁場強度を電子磁気工業社製のガウスメーターを用いて測定した。\qty{1000}{Hz}以下の低周波数帯では、
バイポーラ電源の示す電流値と実際に印加されている電流値に誤差がなかった。よって直流電源を流した時の磁場強度をもとに、目的の振幅に対応する
交流電源を印加した。液体窒素によるサンプルの温度低下は、断熱性に優れた硬質ウレタンフォーム製のカバーでサンプルを覆いながら測定を行った。

\subsection{磁気共鳴イメージング(MRI)}
MRI測定は、東京大学 関野研究室の協力のもと、BioSpec製 70/20USRを用いて測定を行った。MRI測定では、Gd-6の粒径別(\qty{10.5}{nm},\qty{14.1}{nm},\qty{16.9}{nm},\qty{23.5}{nm})
、そして寒天のSpin Echo法による$T_1,T_2$緩和測定を行った。
\subsubsection{MRI評価用ファントムの作製}

\textit{in vitro}でのMRIの緩和率の評価は、造影剤を水に分散させた状態のファントムと呼ばれるものを作ることで行う。ここで磁気ナノ微粒子の凝集を防ぐため、
超音波ホモジナイザーを用いて測定した。PP製の\qty{50}{mL}遠沈管に純水\qty{30}{mL}入れ、金属イオンの濃度が純水に対して\qty{1.0}{mM}となるように秤量した。
純水と測定サンプルを混ぜ超音波ホモジナイザーで20分処理し、純水に懸濁させた。寒天はビーカーに\qty{50}{mL}遠沈管に入れる純水に対して\qty{0.8}{wt\%}で秤量し、マグネティックスタラーを用いて
撹拌しながら、寒天の溶解温度である\qty{90}{\degreeCelsius}まで加温し溶かした。溶かした寒天は遠沈管に入れ、そのまま常温で放置して固化させた。
作製したファントムを図 \ref{fig:ファントム}に示す。
\begin{figure}[htb]
  \centering
   \includegraphics[width=0.4\textwidth]{ファントム.jpg}
  \caption{作成したファントムの写真}\label{fig:ファントム}
\end{figure}

\subsubsection{$T_1$,$T_2$緩和測定}
作製した磁気ナノ微粒子をファントムにしたものの、$T_1$,$T_2$緩和時間を評価した。ファントムはRFコイル内にテープで固定し、MRI装置の撮像部に静置した。
$T_1$緩和測定ではパルス系列のエコー時間(Echo Time)は\qty{2500}{ms}に固定し、繰り返し時間(Reception Time)を\qty{12}{ms}から\qty{48}{ms}間隔で測定を行った。
$T_2$緩和測定ではパルス系列の繰り返し時間は$TR$は\qty{11}{ms}から\qty{11}{ms}間隔に、エコー時間$TE$は\qty{12}{ms}に固定した。
これらにより$T_1$,$T_2$緩和曲線が得られ、緩和率$R_1$,$R_2$を算出した。

\subsection{アミノ基修飾}
アミノ基修飾は(3-アミノプロピル)トリエトキシシラン(APTES)を用いて作製した磁気ナノ微粒子の表面にアミノ基を導入した。ここで作製したサンプルは\ce{SiO2}包含されているが、アミノ基修飾させる上で十分量ではないため、
シリカ前駆体としてテトラエトキシシラン(TEOS)を用いてゾルゲル法であるst\"{o}ber法を応用して、サンプルの表面にシリカ層を形成した。
アミノ基修飾に使用したサンプルはGd-6を用いた。PP製\qty{50}{mL}遠沈管にEtOHを入れ、サンプルを\qty{70}{mg}を混ぜ、ホモジナイザーで30分間超音波処理を行い、
均一に分散させた。PP製\qty{500}{mL}三角フラスコに超音波処理したサンプルとエタノール(EtOH)を\qty{200}{mL}になるように入れた。さらに、\qty{0.0355}{mL} TEOS、\qty{10}{mL} APTES、
触媒として\qty{2}{mL} アンモニア水加えた。混合した溶液はメカニックスターラーで\qty{450}{rpm}、温度を\qty{85}{\degreeCelsius}にし24時間反応させた。
反応後、遠心分離を用いて\qty{3500}{rpm}、5分間を4回行い洗浄し、ニンヒドリン試験を行い遊離$-\ce{NH2}$基の存在がないことを確認した。
得られた沈殿物は\qty{55}{\degreeCelsius}の乾燥炉に入れ2日以上乾燥させた。
\begin{figure}[htbp]
  \centering
  \includegraphics[width=0.7\textwidth]{ニンヒドリン反応_遠心分離.png}
  \caption{遠心分離後のニンヒドリン反応}
  \label{fig:ニンヒドリン反応遠心分離}
\end{figure}
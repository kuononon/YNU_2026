%%%%%%%%% JPS abstract %%%%%%%%%%%%%%%%%%%%%%%%%%%%%%%%%%%%%%%%%%
\documentclass[11pt,a4paper]{jreport}

%%%%%%%%% packages %%%%%%%%%%%%%%%%%%%%%%%%%%%%%%%%%%%%%%%%%%%%%%
\usepackage{masterp}
\usepackage[version=4]{mhchem}
\usepackage{xcolor}
\usepackage{soul}
\usepackage{mathrsfs}
\usepackage{placeins}
\usepackage{makecell}
\newcommand{\jhl}[1]{\hl{\mbox{#1}}}
\newcommand{\Resovist}{Resovist\textsuperscript{\textregistered}}
\renewcommand{\theenumi}{\roman{enumi}}
\renewcommand{\labelenumi}{(\theenumi)}
\renewcommand{\theenumii}{\theenumi-\alph{enumii}}
\renewcommand{\labelenumii}{(\theenumii)}
\hypersetup{setpagesize=false,
bookmarksnumbered=true,
colorlinks=true,
linkcolor=Black,
urlcolor=RoyalBlue,
citecolor=Black
}

%%%%%%%%% header %%%%%%%%%%%%%%%%%%%%%%%%%%%%%%%%%%%%%%%%%%%%%%%%
\begin{document}


\vspace{1pt}

\begin{center}
\vspace{20mm}
{\large\noindent 令和7年度 \\
修士論文}\\
\vspace{60mm}
{\Huge\noindent\textbf{\ce{Gd}ドープ\ce{Mn}-\ce{Zn} ferriteナノ微粒子の}}\\
\medskip
{\Huge\noindent\textbf{磁気特性と磁気粒子イメージング}}\\
\vspace{\baselineskip}
\vspace{80mm}

{\LARGE\noindent
\vspace{\baselineskip}
横浜国立大学大学院\:
理工学府\\
数物・電子情報系専攻\:
物理工学教育分野\\
\vspace{\baselineskip}
一柳研究室\\
24NC216 \qquad
楠本 悠羽\\
}
\vspace{40mm}

\end{center}

\thispagestyle{empty}
\clearpage

%%%%%%%%% main %%%%%%%%%%%%%%%%%%%%%%%%%%%%%%%%%%%%%%%%%%%%%%%%%%

%=====================================================================================

\begin{abstract}

磁性体をナノサイズにすることで特異的な性質が見られ、超常磁性もそのひとつである。
当研究室ではこれらの磁性ナノ微粒子の医療応用に向けた研究を行っており、磁気粒子イメージング(Magnetic Particle Imaging: MPI)もそのひとつである。
本研究においてはMPI応用に向けたトレーサーの開発を目指して研究を行った。
\ce{Mn_{(0.8-\textit{x})}Zn_{0.2}Gd_{\textit{x}}Fe2O4}を、本研究室独自の湿式混合法を用いてGdドープ量、粒径などを変えて作製した。

XRD測定から、作製したサンプルが単相のスピネル構造を持つことを確認し、(440)のピークが低角側にシフトすることから\ce{Gd}のドープを確認した。
また、Gd-6で低角側へのシフトが止まり、Gd-20のドープ量においてはスピネル構造には見られない不純物のピークが確認されたため\ce{Gd}の固溶限はGd-6付近に存在すると考えられる。

SQUID磁束計を用いた磁化測定からは、\ce{Mn}-\ce{Zn} ferriteナノ微粒子への\ce{Gd}ドープ量の増加に伴い飽和磁化$M_S$、初透磁率$\mu_i$が増加することを確認した。

また、サンプルの粒径によっては、$T = \SI{300}{K}$の$M$-$H$ループで負の保磁力を持つ逆ヒステリシスループの挙動を示した。
これは広い粒径分布を持つことで、異方性の大きな粒子と小さな粒子の間に反強磁性的双極子相互作用が働くことに起因する。
さらに、残留磁化$M_r$の温度依存性を確認することにより、転移温度$T_r$以下の温度ではすべての粒子が強磁性化し、磁性相が1つになるために正のヒステリシスループが観測されることを確認した。

\ce{Gd}ドープ量別のサンプルにおけるMPIシグナル測定からは最も大きな初透磁率$\mu_i$を示す、Gd-6のサンプルが最大の応答強度を示した。
Gd-6の粒径別サンプルについては周波数を変えて測定を行い、磁化の大きさと磁気異方性から各周波数帯において最適な粒径が異なることを明らかにした。

また、交流磁場下における動的ヒステリシスループにおいても測定を行った。
この測定からは、動的ヒステリシスループの見かけの保磁力$H_c$や最大磁化$M_\textrm{max}$、非線形性についても調べ、MPIシグナルとの相関を確認した。
また、動的ヒステリシスループをフーリエ変換することで、粒径や周波数によって基本波強度や第三高調波強度がどのように変化するのかについても調べた。
その結果から、駆動磁場周波数別のMPIシグナルにおいて測定を行うことができなかった$f = \SI{48}{kHz}$以降の周波数帯においてはGd-6 (\SI{8.9}{nm})のサンプルが最適であることを明らかにした。
また、駆動磁場強度を変えた際の空間分解能についても調べ、小さい駆動磁場強度において空間分解能が向上することを明らかにした。
\end{abstract}


%=====================================================================================

 \tableofcontents

%=====================================================================================

\input{chapters/1intro}

%=====================================================================================

\input{chapters/2theory}

%=====================================================================================

\input{chapters/3method}

%=====================================================================================

\input{chapters/4resultd}

%=====================================================================================

\input{chapters/5concl}

%=====================================================================================
 %\chapter*{参考文献}

 \vspace{-10mm}

\bibliographystyle{naturemag}
\bibliography{hoge}

\clearpage

\chapter*{謝辞}
本研究は以下の助成により行った。
\begin{itemize}
  \item 基盤研究(A), 「スーパースピングラス磁気ナノ微粒子の創製とナノ・セラティクスの実現」一柳優子, 2020--2024 年採択
  \item 学長戦略「ナノ物性物理とバイオの融合研究拠点」活動支援事業 一柳優子, 2022 年採択
  \item IEEE (The Institute of Electrical and Electronics Engineers, Incorporated) Conference SupportGrant
  \item 徳山科学技術振興財団 2022 年度国際シンポジウム助成
  \item JST 国際強化支援
  \item 東京工業大学 科学技術創成研究院 フロンティア材料研究所 一般共同研究 B No.13「Gdをドープした Ferrite 磁気ナノ微粒子の磁気特性」
  \item 東京工業大学科学技術創成研究院フロンティア材料研究所一般共同研究 B No.21 「鉄系酸化物磁気ナノ微粒子の形状制御と磁気及び熱力学的特性分析」 一柳優子, 2024 年度採択
  \item 阪大 AMED 令和 5 年度 AMED 橋渡し研究プログラム異分野シーズ No.H-48 「セラノスティクス機能を持つ磁気ナノ微粒子の開発」 一柳優子, 2022--2024 年採択
  \item KEK 放射光共同利用実験課題 2021G073「ZnO 系希薄磁性半導への Gd ドープ効果と局所構造解析」 一柳優子, 2022--2024 年採択
  \item KEK 放射光共同利用実験課題 2024G600 「Gd, Zn 共ドープ Mn-Zn ferrite 系ナノ微粒子における金属原子の配位特性と局所構造解析」 一柳優子, 2024--2026 年採択
  \item 日本学術振興会 科学研究費助成事業 基盤研究 (B) 「超常磁性スピンクラスターの磁気緩和現象の解明と創薬への応用」 一柳優子, 2025--2027 年採択
  \item 高橋経済研究財団 研究助成 255 「がん細胞選択性を持つ磁気ナノ微粒子の開発」 一柳優子, 2025年度採択
  \item YNU 国際ネットワークハブ 「ナノ物性物理とバイオの融合研究拠点」 一柳優子, 2024--2026 年採択
\end{itemize}

本研究を行うにあたり、指導教員の一柳優子教授には研究における様々なアドバイスに加え、多くの経験をさせていただき、大変感謝しております。

また、本研究におけるSQUID磁束計を用いた磁化測定につきましては、東京科学大学 総合成研究院 川路均先生、木谷卓先生、大阪大学 熱・エントロピー科学研究センター 中野元裕先生、宮崎裕司先生、中沢康浩先生、横浜市立大学 国際総合科学部 山田重樹先生のご協力のもと測定を行うことができました。
また、動的ヒステリシスループの測定につきましては、茨城工業高等専門学校 国際創造工学科 小野寺礼尚先生、筑波大学 数理物質系物理工学域 喜多英二先生のご協力のもと行うことができ、解析方法につきましてもご指導いただきました。
また、XAFS測定につきましては、高エネルギー加速器研究機構 物質構造科学研究所 阿部仁先生をはじめとする関係者の皆様にご協力いただきました。
さらに、RINT2500を用いたXRD測定につきましては、横浜国立大学 教育人間科学部 津野宏先生にご協力いただきました。
そして、MRI測定におきましては、東京大学 大学院工学系研究科 バイオエンジニアリング専攻 関野正樹先生にご協力いただきました。
各大学、研究機関の先生方にはお忙しい中、時間を割いていただき、時には指導していただき誠にありがとうございました。

また、研究室でともに研究を行った、坂本壮さん、森脇智将さん、藤田陽平さん、新居和音さん、阿部凌大さん、天野広希さん、渡邉将太郎さん、長谷川万理萌さん、葛井遼さん、下釜知也さん、安澤颯介さん、矢野凌大
さん、飯島涼太さん、花田拓海さん、川井楓さん、砂川遼太さん、星川直輝さん、三浦玖遠さんには様々な議論を通して多くのことを学ばせていただきました。
特に、実験をともに行った飯島涼太さん、星川直輝さん、三浦玖遠さん、研究に対しての様々なアドバイスや議論を行った渡邉将太郎さんには大変お世話になりました。

厚く御礼申し上げるとともに、これを謝辞と代えさせていただきます。


\clearpage

\chapter*{業績リスト}
本研究に関連する学会発表は以下の通りである。
\begin{itemize}
  \item $16^{th}$ International Symposium Nanomedicine (ISNM2023), 2023.11.20--22 大阪公立大学杉本キャンパス \\
    講演形式:ポスター, 講演番号:P-2 \\
    発表題目: " Harmonic response and heat dissipation of \ce{Gd3+} Mn ferrite nanoparticles under AC magnetic field " \\
    Y. Kusumoto, T. Sakamoto, K. Nii, Y. Fujita, T. Moriwaki, R. Abe, H. Amano, R. Katui, T. Shimogama, M. Hasegawa, S. Yasuzawa, and Y. Ichiyanagi
  \item 2024年応用物理学会春季学術講演会, 2024.3.22--25 東京都市大世田谷キャンパス \\
    講演形式:ポスター, 講演番号:24p-P05-6 \\
    発表題目: " \ce{Mn0.8Zn0.2Fe2O4}ナノ微粒子の交流磁場下における第三高調波応答と直流磁場の印加方向による抑制効果 " \\
    楠本 悠羽, 坂本 壮,新居 和音,藤田 陽平,森脇 智将,阿部 凌大,天野 広希, 長谷川 万里萌,葛井 遼,下釜 知也,安澤 颯介,一柳 優子
  \item ナノ学会第22回大会, 2024.5.22--24 東北大学青葉山新キャンパス \\
    講演形式:ポスター, 講演番号:P-18 \\
    発表題目: " \ce{Mn0.8Zn0.2Fe2O4}ナノ微粒子の交流磁場下における第三高調波応答と直流磁場の印加方向による抑制効果 " \\
    楠本悠羽, 阿部凌大, 天野広希, 長谷川万里萌, 一柳優子
  \item 第85回応用物理学会秋季学術講演会, 2024.9.16--20 朱鷺メッセ \\
    講演形式:ポスター, 講演番号:20p-P02-6 \\
    発表題目: " MnZnFe2O4ナノ微粒子の第三高調波応答におけるGdドープの効果と周波数依存性 " \\
    楠本悠羽, 飯島涼太, 阿部凌大, 天野広希, 長谷川万里萌, 渡邉将太郎, 一柳優子
  \item 第60回熱測定討論会, 2024.9.26--28 京都府立大学下鴨キャンパス \\
    講演形式:ポスター, 講演番号:P2-10 \\
    発表題目: " GdドープMn-Zn ferriteの交流磁場下における第三高調波応答と昇温効果 " \\
    楠本悠羽, 飯島涼太, 阿部凌大, 天野広希, 長谷川万里萌, 渡邉将太郎, 一柳優子
  \item 第72回応用物理学会春季学術講演会, 2025.3.14--17 東京理科大学野田キャンパス \\
    講演形式:ポスター, 講演番号:17a-P01-5 \\
    発表題目: " GdドープMn-Zn ferriteナノ微粒子のXAFSによる局所構造解析 " \\
    楠本悠羽, 飯島涼太, 阿部凌大, 天野広希, 渡邉将太郎, 一柳優子
  \item 第23回ナノ学会大会, 2025.5.14--16 東京都立大学 \\
    講演形式:ポスター, 講演番号:P-03 \\
    発表題目: " 磁気粒子イメージング(MPI)を目指した\ce{Gd}ドープ\ce{Mn}-\ce{Zn} ferriteの作製 " \\
    楠本悠羽, 渡邉将太郎, 一柳優子
  \item 第86回応用物理学会秋季学術講演会, 2025.9.7--10 名城大学 天白キャンパス \\
    講演形式:口頭発表, 講演番号:8a-N303-2 \\
    発表題目: " セラノスティクス応用に向けたGdドープMn-Zn ferriteナノ微粒子の作製と物性評価 " \\
    楠本悠羽, 星川直輝, 三浦玖遠, 渡邊将太郎, 一柳優子
  \item 第61回熱測定討論会, 2025.9.24--26 横浜国立大学 \\
    講演形式:ポスター, 講演番号:P1-4 \\
    発表題目: " 冷却ピックアップコイルを用いたGdドープMn-Zn ferriteの第三高調波測定 " \\
    楠本悠羽, 星川直輝, 三浦玖遠, 渡邊将太郎, 一柳優子
  \item 2.	Material Research Meeting 2025, 2025.12.8--13 パシフィコ横浜ノース \\
    講演形式:ポスター, 講演番号:H2-P103-01 \\
    発表題目: " Effect of Gd Doping on Mn-Zn Ferrite Nanoparticles for MPI Applications " \\
    Yuu Kusumoto, Naoki Hoshikawa, Kuon Miura, Shotaro Watanabe, Yuko Ichiyanagi
\end{itemize}

本研究に関する学術論文は以下の通りである。
\begin{enumerate}
  \item S.~Watanabe, R.~Yano, H.~Amano, R.~Abe, \textbf{Y.~Kusumoto}, M.~Hasegawa, Y.~Ichiyanagi,
  ``Improvement of Water Dispersion of Magnetic Nanoparticles and Their Effect on Magnetic Relaxation,''
  \textit{e-Journal of Surface Science and Nanotechnology}, \textbf{23}(2), 97--107 (2025).
  doi:\,10.1380/ejssnt.2025-015.

  \item H.~Amano, R.~Abe, S.~Watanabe, \textbf{Y.~Kusumoto}, Y.~Ichiyanagi,
  ``Preparation of Ce$^{3+}$ doped ZnO nanoparticles via a wet chemical method and analysis of their local structure,''
  \textit{Phys.\ Chem.\ Chem.\ Phys.}, \textbf{27}, 10499--10505 (2025).
  doi:\,10.1039/D5CP00124B.
\end{enumerate}
\clearpage


\end{document}

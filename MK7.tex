\section{謝辞}
本研究を進めるにあたり、様々なご指導をいただきました一柳優子教授に深く感謝申し上げます。 

また、研究室で共に研究を進めてきた楠本悠羽さん、渡邉将太郎さん、川井楓さん、砂川遼太さん、星川直輝さんには、多くの議論交わし、様々な助言や励ましをいただきました。心より感謝申し上げます。特に、実験や測定のサポート、様々な助言を多くしていただきました楠本悠羽さん、共に議論を交わし研究を進めた星川直輝さん、実験や測定のサポートをして頂きました渡邉将太郎さん、重ねて感謝申し上げます。

また本研究において、RINT2500を用いたXRD測定につきましては、横浜国立大学 教育人間学部 津野宏先生にご協力いただきました。XAFS測定につきましては、高エネルギー加速器研究機構 物質構造科学研究所 関係者の皆様にご協力いただきました。MRI測定につきましては、東京大学 大学院工学系研究科 バイオエンジニアリング先行 関野正樹先生にご協力いただきました。SQUID磁束計を用いた磁化測定につきましては、東京科学大学 総合研究院 川路均先生、木谷卓先生、大阪大学 熱 エントロピー科学研究センター 中野元裕先生、宮崎裕司先生、中沢康浩先生、横浜市立大学 国際総合学部 山田重樹先生にご協力いただきました。各大学、研究機関の関係者の皆様に深く感謝申し上げます。

厚く御礼申し上げるとともに、これを謝辞と代えさせていただきます。

\subsection{本研究に関わる研究費助成一覧}
また、本研究は以下の研究費助成を受けて実施されました。

\begin{itemize}
  \renewcommand{\labelitemi}{・}
  \item KEK 放射光共同利用実験課題 2024G600
  「Gd, Zn 共ドープ Mn-Zn ferrite 系ナノ微粒子における金属原子の配位特性と局所構造解析」
  一柳優子,2024--2026 年採択
  \item 日本学術振興会 科学研究費助成事業 基盤研究 (B)
  「超常磁性スピンクラスターの磁気緩和現象の解明と創薬への応用」
  一柳優子,2025--2027 年採択
  \item 高橋経済研究財団 研究助成 255
  「がん細胞選択性を持つ磁気ナノ微粒子の開発」
  一柳優子,2025 年度採択
  \item YNU 国際ネットワークハブ
  「ナノ物性物理とバイオの融合研究拠点」
  一柳優子,2024--2026 年採択
\end{itemize}

\section{業績}
\begin{itemize}
  \renewcommand{\labelitemi}{・}
  \item 18th International Sinposium on Nanomedicine (ISNM2025) (2025.12.1-3広島大学広仁会館)
ポスター番号 P-01
「Optimization of Magnetic Properties and MPI signals of Gd-Doped Mn-Zn Ferrite Nanoparticles」
K. Miura, Y. Kusumoto, N. Hoshikawa, S. Watanabe, and Y. Ichiyanagi
\end{itemize}

